\documentclass[11pt,a4paper]{article}
\usepackage[a4paper, total={6.5in, 9.5in}]{geometry}
\usepackage[ngerman]{babel}
\usepackage{hyphenat}
\usepackage{lmodern}
\usepackage{parskip}
\usepackage{enumitem}
\usepackage{sectsty}
\sectionfont{\centering}
\subsectionfont{\centering}

\author{FC Wacker Innsbruck}
\title{Statuten des FC Wacker Innsbruck}
\begin{document}
\setlength{\parskip}{25pt}
\setcounter{secnumdepth}{0}
\centering

\textbf{\huge VEREINSSTATUTEN}

\textbf{des}

\textbf{\huge FC WACKER INNSBRUCK}

\textbf{(ZVR-Zahl: 180936836)}

\vspace{100ex}

\textbf{Fassung gemäß Beschluss der Generalversammlung vom 01.03.2022}

\clearpage
\flushleft
\setlength{\parskip}{10pt}

\tableofcontents

\clearpage

\section{I. ALLGEMEINE BESTIMMUNGEN}

\subsection{§ 1
Name und Sitz}

(1)
Der Verein führt den Namen "`FC Wacker Innsbruck"' und ist im Zentralen Vereinsregister des Bundesministeriums für Inneres zu ZVR-Zahl 180936836 eingetragen.

(2)
Der Verein hat seinen Sitz in Innsbruck.

\subsection{§ 2
Vereinsfarben und Vereinswappen}

(1)
Die Vereinsfarben sind "`Schwarz-Grün"'.

(2)
Das Vereinswappen (Logo) visualisiert den Ursprung, die Identität und die Zukunft des Vereins FC Wacker Innsbruck.
Es ist an das Ursprungswappen bei der Vereinsgründung angelehnt und eingebettet in das schwarz-grüne Selbstverständnis:

\subsection{§ 3
Vereinszweck}

(1)
Der Verein, dessen Tätigkeit nicht auf Gewinn gerichtet ist, wird auf gemeinnütziger Basis im Sinne der Bundesabgabenordnung, von einem gewählten Vorstand ehrenamtlich geführt.

(2)
Zweck des Vereins sind die Pflege, Förderung und Verbreitung sportlicher Aktivitäten, insbesondere des Fußballsports, mit allen damit unmittelbar und mittelbar im Zusammenhang stehenden Aufgaben.
Dazu gehört auch das Vertreten und Fördern der Idee, dass Fußball als verbindende Kraft zwischen Kulturen, Nationalitäten, Konfessionen, Menschen unterschiedlicher Weltanschauung oder sexueller Orientierung sowie Menschen mit und ohne Behinderung wirken kann.

(3)
In der Überzeugung, dass regelmäßige sportliche Betätigung - gerade auch bei Jugendlichen - einen persönlichen Zugewinn in physischer und psychischer Hinsicht bewirkt, fördert der Verein sportliche Aktivitäten generell und den Mannschaftssport im Besonderen.
Dadurch sollen die Persönlichkeitsentwicklung und die Verfassung insbesondere von Kindern, Jugendlichen und jungen Erwachsenen positiv beeinflusst und soziale Verhaltensmuster und Werte wie Respekt, Solidarität, Inklusion und Fairness vermittelt und gefestigt werden.

(4)
Der Verein ist politisch, weltanschaulich und konfessionell neutral.
Weltanschauliche, konfessionelle und parteipolitische Ziele und Zwecke dürfen nicht verfolgt oder gefördert werden.

(5)
Der Verein und seine Mitglieder verpflichten sich, jeglichem diskriminierenden Verhalten aufgrund Geschlecht, sexueller Orientierung, Behinderung, Herkunft, Hautfarbe sowie ethnischer oder religiöser Zugehörigkeit bzw. Nichtzugehörigkeit, im Stadion und im Verein entschieden entgegenzutreten, zur Förderung von gesellschaftlich benachteiligten oder unterrepräsentierten Personengruppen im Verein und zur Zusammenarbeit mit allen Organisationen, welche dem Problem der Diskriminierung aus oben genannten Gründen im Fußball entgegenwirken.
Insbesondere Aktivitäten im Rahmen des Netzwerkes "`FARE - Football Against Racism in Europe"' und der österreichischen Kampagne "`FairPlay - Viele Farben. Ein Spiel"' sollen unterstützt werden.
Außerdem werden die Grundsätze des Zehn Punkte Plans der UEFA zur Bekämpfung des Rassismus im Fußball ausdrücklich anerkannt.

\subsection{§ 4
Mittel zur Erreichung des Vereinszweckes}

(1)
Der Vereinszweck soll insbesondere durch die in den Absätzen 2 und 3 angeführten ideellen und materiellen Mittel erreicht werden.

(2)
Als \textbf{ideelle Mittel} dienen

\begin{enumerate}[label=\alph*)]
\item
geregelte Ausbildung der Mitglieder in einzelnen Sportzweigen, insbesondere die Pflege des Fußballsportes auf allen Gebieten des Breiten-, Spitzen- und Gesundheitssportes;
\item
Organisation und Betrieb von Fußballmannschaften im Nachwuchs-, Damen und Amateurbereich;
\item
Aufbau, Organisation, Betrieb und Förderung von Fußballmannschaften im Bereich Behindertensport (z.B. Blindenfußball, E-Rolli-Fußball, SpecialNeeds-Team usw.)
\item
Teilnahme an und Veranstaltung von regionalen, nationalen und internationalen sportlichen Wettkämpfen;
\item
Aufbau, Übernahme oder Kooperationsbetrieb einer Akademie mit angeschlossener Sport- und Trainingsanlage;
\item
Errichtung und fachgemäße Leitung von Sportanlagen;
\item
Beteiligung an wirtschaftlichen Unternehmungen und Gesellschaften aller Art;
\item
regelmäßige gesellschaftliche Versammlungen der Mitglieder in angemessener Anzahl (mindestens ein Vereinsabend pro Quartal) und sonstige Maßnahmen der Mitglieder- und sozialer Fanbetreuung;
\item
Vorträge, Mitgliederkompetenzbildung, Versammlungen, Symposien und Diskussionsforen;
\item
die Herausgabe periodischer oder nicht periodischer Druckschriften und Publikationen aller Art (z.B. Stadionzeitung, usw.) in analoger oder digitaler Form;
\item
Information der Öffentlichkeit über Vereinsaktivitäten (z.B. Homepage, Social Media, usw.);
\item
Aufbau eines Museums, eines Dokumentationszentrums mit angeschlossener Fachbibliothek sowie in weiterer Folge eines wissenschaftlichen Think-Tanks;
\end{enumerate}

(3)
Die erforderlichen \textbf{materiellen Mittel} des Vereins sollen aufgebracht werden durch

\begin{enumerate}[label=\alph*)]
\item
Mitgliedsbeiträge und sonstige Förderungsbeiträge;
\item
Erlöse aus der Organisation von Veranstaltungen oder der Teilnahme an Veranstaltungen jeglicher Art;
\item
Erträgnisse aus Werbung, Vermarktung, Publikationen, Sponsorbeiträgen, Subventionen, Spenden, Zuschüssen, Erbschaften, Vermächtnissen und sonstigen Zuwendungen;
\item
Erträgnisse aus der Zentralvermarktung der ÖFBL bzw. des ÖFB;
\item
Erträgnisse sonstiger finanzieller Verwertung (z.B. Vermietung bzw. Verpachtung von Anlagevermögen, Lizenzierung von Namensrechten und sonstigen Immaterialgüterrechten, Weitergabe von Teilnahmerechten am Spielbetrieb von Fußballligen, usw.)
\item
Erträgnisse aus Gesellschaftsbeteiligungen;
\item
Einnahmen aus der Ausübung des Gastronomiegewerbes und dem Verkauf von Merchandisingartikeln;
\item
Einnahmen aus dem Verkauf bzw. der Vermarktung von dem Verein gehörenden materiellen und immateriellen Rechten;
\item
Transfererlöse;
\item
sonstige Einnahmen;
\end{enumerate}

\subsection{§ 5
Verwendung des Vereinsvermögens}

(1)
Das Vereinsvermögen darf nur für die in den Statuten angeführten, gemeinnützigen Vereinszwecke verwendet werden.

(2)
Die Mitglieder des Vereines dürfen keine Gewinnanteile und in ihrer Eigenschaft als Mitglieder auch keine sonstigen Zuwendungen aus Mitteln des Vereinsvermögens erhalten.

(3)
Es darf keine Person durch Ausgaben, die dem Zweck des Vereines fremd sind, oder durch unverhältnismäßig hohe Vergütungen begünstigt werden.

\subsection{§ 6
Dauer des Vereins}

Die Dauer des Vereins ist unbeschränkt und dauert das Vereinsjahr jeweils vom 01. Juli eines Jahres bis zum 30. Juni des darauffolgenden Jahres.

\section{II. MITGLIEDSCHAFT}

\subsection{§ 7
Arten der Mitgliedschaft}

(1)
Die Mitglieder des Vereins gliedern sich in:

\begin{itemize}
\item
Ordentliche Mitglieder
\item
Fördernde Mitglieder
\item
Kindermitglieder
\item
Familienmitglieder
\item
Ehrenmitglieder
\item
Ehrenobmänner bzw. Ehrenpräsidenten
\end{itemize}

(2)
Ordentliche Mitglieder

Ordentliche Mitglieder sind jene Mitglieder, die vom Vorstand aufgrund einer Beitrittserklärung als solche aufgenommen werden und sich durch Ausübung ihres Stimmrechtes in der Generalversammlung aktiv am Vereinsleben im Rahmen der Vereinsstatuten beteiligen ("`Wackere Mitgliedschaft"').

Ordentliche Mitglieder haben einen jährlichen Mitgliedsbeitrag zu entrichten, dessen Höhe durch einen Beschluss der Generalversammlung festgelegt wird.
Schüler, Lehrlinge, Studenten bis zur Vollendung des 24. Lebensjahres (25. Geburtstag), Pensionisten und Menschen mit Handicap kommen in den Genuss eines ermäßigten Mitgliedsbeitrages, dessen Höhe ebenfalls durch einen Beschluss der Generalversammlung festgelegt wird.

Ab dem Inkrafttreten (§ 28) vorliegender Statuten müssen ordentliche Mitglieder mindestens ein Jahr ununterbrochen Mitglied sein, damit sie bei Vorstandswahlen und Beschlussfassungen in Generalversammlungen über eine (1) Stimme verfügen.
Das Erfordernis der ununterbrochenen, einjährigen Mitgliedschaft ist auch dann erfüllt, wenn diese entweder zur Gänze oder in Teilen fördernder Natur gewesen ist.

(3)
Fördernde Mitglieder

Fördernde Mitglieder sind jene Mitglieder, die vom Vorstand aufgrund einer Beitrittserklärung als solche aufgenommen werden und das Vereinsleben zwar nicht aktiv mitgestalten, aber die Zwecke des Vereins durch Zahlung eines jährlichen Mitgliedsbeitrages, dessen Höhe durch einen Beschluss der Generalversammlung festgelegt wird, oder durch sonstige Zuwendungen jeglicher Art fördern.

Fördernde Mitglieder verfügen über kein Stimmrecht bei Vorstandswahlen und Beschlussfassungen in Generalversammlungen.

(4)
Kindermitglieder

Kindermitglieder sind natürliche Personen ab Vollendung des 6. Lebensjahres (7. Geburtstag) bis zur Vollendung des 14. Lebensjahres (15. Geburtstag), die vom Vorstand aufgrund einer Beitrittserklärung als solche aufgenommen werden ("`Wacker Ma(n)dl"').
Kinder unter 7 Jahren können aus gesetzlichen Gründen nicht Mitglied eines Vereins werden.

Die Beitrittserklärung hat durch den jeweiligen gesetzlichen Vertreter im Namen des Kindes zu erfolgen.
Die Höhe des jährlichen Mitgliedsbeitrages wird durch einen Beschluss der Generalversammlung festgelegt.

(5)
Familienmitglieder

Familienmitglieder sind natürliche Personen, die im Paket 2 ordentliche Mitgliedschaften und bis zu 4 Kindermitgliedschaften mit allen damit verbundenen Rechten und Pflichten zu einem ermäßigten, jährlichen Mitgliedsbeitrag erwerben, dessen Höhe durch einen Beschluss der Generalversammlung festgelegt wird.

Die Beitrittserklärung hat die Namen sämtlicher Personen zu enthalten, die von der Familienmitgliedschaft erfasst sind.
Über den Beitritt entscheidet der Vorstand.

(6)
Ehrenmitglieder

Ehrenmitglieder sind natürliche Personen, die wegen besonderer Verdienste um den Verein über Vorschlag des Vorstands durch Beschluss der Generalversammlung zu solchen ernannt werden.

Ehrenmitglieder sind von der Bezahlung eines Mitgliedsbeitrages befreit.
Ehrenmitglieder verfügen bei Vorstandswahlen und Beschlussfassungen in Generalversammlungen über eine ( 1) Stimme.

(7)
Ehrenobmänner bzw. Ehrenpräsidenten

Ehrenobmänner bzw. Ehrenpräsidenten sind natürliche Personen, die wegen besonderer Verdienste in ihrer Funktion als Obmann bzw. Präsident des Vereins am Ende ihrer Funktionsperiode über Vorschlag des Vorstands durch Beschluss der Generalversammlung zu solchen ernannt werden.
Ehrenobmänner bzw. Ehrenpräsidenten sind von der Bezahlung eines Mitgliedsbeitrages befreit.

Ehrenobmänner bzw. Ehrenpräsidenten verfügen bei Vorstandswahlen und Beschlussfassungen in Generalversammlungen über eine (1) Stimme.

\subsection{§ 8
Erwerb der Mitgliedschaft}

(1)
Mitglieder des Vereins können nur natürliche Personen werden.

(2)
Über die Aufnahme von Mitgliedern entscheidet der Vorstand aufgrund einer Beitrittserklärung mit einfacher Mehrheit.
Die Ablehnung einer Aufnahme kann ohne Angabe von Gründen erfolgen.
Ein Beitritt wird wirksam, wenn er nicht binnen 3 Monaten nach Einlangen der Beitrittserklärung vom Vorstand schriftlich abgelehnt wird.

(3)
Gegen die Ablehnung einer Aufnahme steht dem Betroffenen das Recht zu, längstens binnen 14 (vierzehn) Tagen einen Einspruch an die Generalversammlung zu erstatten.
In einem solchen Fall entscheiden die Mitglieder der Generalversammlung in der nächsten ordentlichen Generalversammlung mit einfacher Mehrheit über die Aufnahme des Betroffenen als Mitglied.
Ein Beschluss der Generalversammlung über die Ablehnung ist nicht anfechtbar.

(4)
Ist der Mitgliedswerber noch nicht volljährig, so bedarf die Beitrittserklärung der Zustimmung seines gesetzlichen Vertreters.

\subsection{§ 9
Rechte der Mitglieder}

(1)
Alle Mitglieder sind berechtigt, im Rahmen der Statuten am Vereinsleben teilzunehmen und die Einrichtungen des Vereins im vom Vorstand gestatteten Ausmaß zu beanspruchen.

(2)
Nur ordentliche Mitglieder, Ehrenmitglieder und Ehrenobmänner bzw.
Ehrenpräsidenten, die bereits volljährig sind und deren Mitgliedschaft zudem seit mindestens einem Jahr ununterbrochen besteht, haben das aktive und passive Wahlrecht sowie das Stimmrecht in der Generalversammlung.
Fördernden Mitgliedern kommt weder ein aktives noch ein passives Wahlrecht und auch kein Stimmrecht in der Generalversammlung zu.

(3)
Das aktive und passive Wahlrecht ruht für jene Mitglieder, die als Spieler für den Verein registriert sind.

(4)
Jedes Mitglied ist berechtigt, vom Vorstand die Ausfolgung der Statuten zu verlangen.

(5)
Mindestens ein Zehntel der Mitglieder (Kopfmehrheit) kann vom Vorstand die Einberufung einer Generalversammlung verlangen.

(6)
Die Mitglieder sind in jeder ordentlichen Generalversammlung vom Vorstand über die Tätigkeit und finanzielle Gebarung des Vereins zu informieren.
Wenn mindestens ein Zehntel der Mitglieder (Kopfmehrheit) dies unter Angabe von Gründen verlangt, hat der Vorstand den betreffenden Mitgliedern eine solche Information auch sonst binnen 4 (vier) Wochen zu geben.

(7)
Die Mitglieder sind vom Vorstand über den geprüften Rechnungsabschluss (Rechnungslegung) zu informieren.
Geschieht dies in der Generalversammlung sind die Rechnungsprüfer einzubinden.

\subsection{§ 10
Pflichten der Mitglieder}

(1)
Alle Mitglieder sind verpflichtet, die Interessen des Vereins nach besten Kräften zu fördern, seine Werte zu achten und alles zu unterlassen, wodurch Ansehen und Zweck des Vereins leiden könnten.

(2)
Alle Mitglieder sind verpflichtet, die Statuten und ihre Durchführungsbestimmungen sowie die Beschlüsse der Vereinsorgane genau zu beachten.

(3)
Alle Mitglieder sind zur pünktlichen Bezahlung der Mitgliedsbeiträge verpflichtet.
Die jährlichen Mitgliedsbeiträge sind jeweils im Vorhinein bis spätestens 30. Juni eines jeden Jahres zu entrichten.

\subsection{§ 11
Beendigung der Mitgliedschaft}

(1)
Die Mitgliedschaft erlischt durch Tod, freiwilligen Austritt, Streichung, Ausschluss, Aberkennung oder Auflösung des Vereins.

(2)
Tod

Nachdem eine Mitgliedschaft beim Verein höchstpersönlich ist, erlischt diese bei natürlichen Personen durch Tod.

(3)
Freiwilliger Austritt

Der freiwillige Austritt von ordentlichen Mitgliedern, fördernden Mitgliedern, Kinderund Familienmitgliedern kann nur zum 30. Juni jeden Jahres erfolgen, wobei dem Vorstand des Vereins bis längstens 1 . Juni desselben Jahres eine schriftliche Austrittserklärung zugegangen sein muss.
Der Austritt entbindet nicht von der Erfüllung der bis zum Austrittszeitpunkt entstandenen Verbindlichkeiten gegenüber dem Verein.

(4)
Streichung

Die Streichung von ordentlichen Mitgliedern, fördernden Mitgliedern, Kinder- und Familienmitgliedern kann der Vorstand dann vornehmen, wenn dieses den Mitgliedsbeitrag trotz Mahnung nicht bis spätestens 01.12. des laufenden Geschäftsjahres einbezahlt hat.

Gegen diese Streichung steht dem betroffenen Mitglied innerhalb einer Frist von längstens 14 (vierzehn) Tagen nach dessen Bekanntgabe das Recht des schriftlichen Einspruchs an den Vorstand zu.
Im Falle eines Einspruchs gegen die Streichung hat der Vorstand das Schiedsgericht nach § 24 einzuberufen, bis zu dessen Entscheidung die Mitgliedsrechte ruhen.
Gegen die Entscheidung des Schiedsgerichts ist kein Rechtsmittel möglich.

Die Streichung entbindet nicht von der Erfüllung der bis zum Streichungszeitpunkt entstandenen Verbindlichkeiten gegenüber dem Verein.

(5)
Ausschluss

Der Ausschluss eines Mitgliedes aus dem Verein kann vom Vorstand mittels Beschluss wegen grober Verletzung der Mitgliedspflichten bzw. der Amtspflichten eines Funktionärs oder wegen unehrenhaften bzw. vereinsschädigenden Verhaltens verfügt werden.
Ein Ausschluss kann auch im Falle der Nichtunterwerfung unter das Schiedsgericht oder der Nichtanerkennung einer Entscheidung desselben erfolgen.
Sofern ein Mitglied des Vorstands ausgeschlossen werden soll, kommt diesem Mitglied bei der Beschlussfassung kein Stimmrecht zu.

Gegen den Ausschluss steht dem betroffenen Mitglied innerhalb einer Frist von längstens 14 (vierzehn) Tagen nach dessen Bekanntgabe das Recht des schriftlichen Einspruchs an den Vorstand zu.
Im Falle eines Einspruchs gegen den Ausschluss hat der Vorstand das Schiedsgericht nach § 24 einzuberufen, bis zu dessen Entscheidung die Mitgliedsrechte ruhen.
Gegen die Entscheidung des Schiedsgerichts ist kein Rechtsmittel möglich.

Der Ausschluss entbindet nicht von der Erfüllung der bis zum Ausschlusszeitpunkt entstandenen Verbindlichkeiten gegenüber dem Verein.

(6)
Aberkennung

Die Aberkennung der Ehrenmitgliedschaft bzw. des Titels als Ehrenobmann bzw. Ehrenpräsident kann aus den in Abs. 5 genannten Gründen von der Generalversammlung über Antrag des Vorstands beschlossen werden.

(7)
Auflösung des Vereins

Mit einer Auflösung des Vereins geht selbstverständlich auch der Verlust der Mitgliedschaft an diesem Verein einher.

(8)
Im Falle der Beendigung der Mitgliedschaft aus welchem Grund auch immer hat das jeweilige Mitglied weder Anspruch auf Rückerstattung von Beitrittsgebühren oder Mitgliedsbeiträgen noch auf das Vereinsvermögen.

\section{III. ORGANE}

\subsection{§ 12
Organe des Vereins}

(1)
Die Organe des Vereins sind:

\begin{enumerate}[label=\alph*)]
\item
die Generalversammlung
\item
der Vorstand
\item
der Beirat
\item
das Schiedsgericht
\item
die Rechnungsprüfer bzw. Abschlussprüfer
\end{enumerate}

(2)
Jedes Organ handelt nach Maßgabe der ihm in diesen Statuten zugewiesenen Zuständigkeiten und Wirkungsweisen sowie der geltenden Gesetze.

(3)
In die in Abs 1 lit a - d genannten Vereinsorgane können nur Mitglieder gewählt oder berufen werden.
Wiederwahl oder wiederholte Berufung sind zulässig.

(4)
Mitglieder von Kontroll-, Geschäftsführungs- und Vertretungsorgangen anderer Vereine oder deren Tochtergesellschaften in der österreichischen FußballBundesliga dürfen keine Funktionen in Organen des Vereins übernehmen.

\subsection{§ 13
Die Generalversammlung}

(1)
Die Generalversammlung ist die Mitgliederversammlung im Sinne des Vereinsgesetzes 2002 und somit das oberste Vereinsorgan.

(2)
Eine ordentliche Generalversammlung findet einmal jährlich innerhalb von längstens 8 (acht) Monaten nach Beendigung des Geschäftsjahres statt.
Die ordentliche Generalversammlung wird durch den Vorstand einberufen.

(3)
Eine außerordentliche Generalversammlung ist vom Vorstand auf

\begin{enumerate}[label=\alph*)]
\item
Beschluss des Vorstands oder der ordentlichen Generalversammlung,
\item
Verlangen der Rechnungsprüfer bzw. des Abschlussprüfers, oder
\item
Verlangen von mindestens einem Zehntel der Mitglieder
\end{enumerate}

einzuberufen.
Ein Verlangen auf eine außerordentliche Generalversammlung hat gleichzeitig jene(n) Antrag/Anträge zu enthalten, die bei der außerordentlichen Generalversammlung auf die Tagesordnung gebracht werden sollen.
Die Antragspunkte müssen der Beschlussfassung der Generalversammlung unterliegen, andernfalls gilt der Antrag als nicht gestellt.
Ein Antrag ohne Angabe von Gründen ist nichtig.

(4)
Daneben ist eine außerordentliche Generalversammlung in den Fällen der §§ 17 Abs 5 erster Satz, 20 Abs 3 lit d von den Rechnungsprüfern bzw. dem Abschlussprüfer und im Falle des § 17 Abs 5 zweiter Satz vom gerichtlich bestellten Kurator einzuberufen.

(5)
Die außerordentliche Generalversammlung hat längstens binnen 6 (sechs) Wochen ab dem Zeitpunkt des Beschlusses bzw. des Einlangens des Verlangens beim Vorstand stattzufinden.

(6)
Sowohl zu den ordentlichen als auch zu den außerordentlichen Generalversammlungen sind alle teilnahmeberechtigten Mitglieder mindestens 3 (drei) Wochen vor dem Termin schriftlich einzuladen, wobei für die Rechtzeitigkeit der Einladung das Datum des Poststempels maßgebend ist.
Die Einladung kann per E-Mail oder per Telefax erfolgen, wenn seitens des Mitglieds eine E-MailAdresse oder eine Fax-Nummer hinterlegt wurde, ansonsten per Post.

Zusätzlich hat eine Einladung durch Veröffentlichung auf der offiziellen Homepage des Vereins stattzufinden.

Die Einladung hat den Zeitpunkt, den Ort und die vorläufige Tagesordnung zu enthalten und erfolgt durch den Vorstand, durch die Rechnungs- bzw. den Abschlussprüfer, oder durch einen gerichtlich bestellten Kurator.

(7)
Verstreicht trotz ordnungsgemäßem Verlangen auf Einberufung einer außerordentlichen Generalversammlung die Frist, ohne dass der Vorstand entsprechend tätig wird, so sind diejenigen, die das Verlangen ordnungsgemäß beim Vorstand eingebracht haben, berechtigt, unter Wahrung der vorgenannten Formen und Fristen die außerordentliche Generalversammlung selbst auf Kosten des Vereins einzuberufen.

(8)
Jedes Mitglied hat die Möglichkeit, beim Vorstand bis längstens 14 (vierzehn) Tage vor dem Termin der Generalversammlung schriftlich (E-Mail, Telefax oder Post) Anträge einzubringen.
Später eingebrachte Anträge können nur dann in die Tagesordnung aufgenommen werden, wenn sie dem Vorstand vor Beginn der Generalversammlung schriftlich vorliegen und dem Antrag durch Beschluss des Vorstands oder mit einer qualifizierten Mehrheit von zwei Dritteln der anwesenden, stimmberechtigten Mitglieder die Dringlichkeit zuerkannt wird.

(9)
Die Mitglieder müssen spätestens 10 (zehn) Tage vor der Generalversammlung durch Veröffentlichung auf der offiziellen Homepage des Vereins über die Tagesordnungspunkte informiert werden.

(10)
Anträge auf Statutenänderung müssen mit dem Wortlaut der vorgeschlagenen Änderung mit der Tagesordnung bekannt gegeben werden.

(11)
Bei der Generalversammlung sind nur Vereinsmitglieder teilnahmeberechtigt.
Die Übertragung des Stimmrechts auf ein anderes Mitglied im Wege einer schriftlichen Bevollmächtigung ist nicht zulässig.

(12)
Den Vorsitz in der Generalversammlung führt der Präsident, in dessen Verhinderung der Vizepräsident.
Sollte auch dieser verhindert sein, leitet das an Jahren älteste Vorstandsmitglied die Generalversammlung.

(13)
Generalversammlungen haben am Sitz des Vereins stattzufinden.

\subsection{§ 14
Aufgabenkreis der Generalversammlung}

Der Generalversammlung als oberstem Vereinsorgan sind folgende Aufgaben vorbehalten:

\begin{enumerate}[label=\alph*)]
\item
Wahl der Mitglieder des Vorstands und der Rechnungs- bzw. des Abschlussprüfers;
\item
Enthebung des gesamten Vorstands, einzelner Vorstandsmitglieder sowie der Rechnungs- bzw. des Abschlussprüfers;
\item
einmalige Verlängerung der Funktionsperiode des Vorstands bis 2 (zwei) Monate nach Abschluss des laufenden Geschäftsjahres gem. § 17 Abs 7;
\item
Genehmigung der Kooptierung von Vorstandsmitgliedern;
\item
Genehmigung von Rechtsgeschäften zwischen Rechnungs- bzw. Abschlussprüfern und Verein;
\item
Entlastung des Vorstands;
\item
Festsetzung der Höhe der Mitgliedsbeiträge für ordentliche Mitglieder, fördernde Mitglieder, Kindermitglieder und Familienmitglieder;
\item
Beschlussfassung über Einsprüche gegen die Ablehnung der Mitgliedschaft;
\item
Beschlussfassung über Statutenänderungen und die freiwillige Auflösung des Vereins;
\item
Verleihung und Aberkennung der Ehrenmitgliedschaft sowie des Titels eines Ehrenobmanns bzw. Ehrenpräsidenten;
\item
Beschlussfassung über Anträge auf geheime Abstimmung;
\item
Beschluss über die Aufnahme eines Antrags in die Tagesordnung einer Generalversammlung gem. § 13 Abs 8 zweiter Satz;
\item
Beschlussfassung über sonstige von der Tagesordnung umfasste Anträge
\item
Beschlussfassung über alle Maßnahmen, die geeignet sind, die Befugnisse der Generalversammlung zu beeinträchtigen - sohin insbesondere die Überführung der Profiabteilung in eine eigene Rechtsform, die Lizenzübertragung zur Teilnahme an österreichischen Ligen oder die Veräußerung von Anteilen (insbesondere Verringerung der Stammeinlage oder eines sonstigen nominellen Kapitalanteiles) an jener juristischen Person, auf die die Profiabteilung übertragen wird bzw. wurde, sowie die Übertragung von Stimmrechten als Gesellschafter dieser juristische Person.
Die Beschlussfassung darüber erfordert die Mehrheit von zwei Drittel der gültigen Stimmen.
\end{enumerate}

\subsection{§ 15
Beschlüsse}

(1)
Die Generalversammlung ist unabhängig von der Anzahl der bei ihrem Beginn anwesenden Mitglieder beschlussfähig.

(2)
Die Generalversammlung fasst ihre Beschlüsse grundsätzlich mit einfacher Mehrheit der abgegebenen, gültigen Stimmen, sofern diese Statuten nicht ausdrücklich eine qualifizierte Mehrheiten von zwei Dritteln i.S.d. § 16 vorsehen.
Bei Stimmengleichheit gilt der Antrag als abgelehnt.

(3)
Die Abstimmung in der Generalversammlung erfolgt grundsätzlich offen mit Handzeichen.
Wird ein Antrag auf geheime Abstimmung eingebracht, so entscheidet die Generalversammlung in offener Abstimmung mit einfacher Mehrheit der abgegebenen, gültigen Stimmen über diesen Antrag.
Über Beschlüsse gemäß § 16 lit. a-c ist jedenfalls geheim abzustimmen.

(4)
Gültige Beschlüsse können nur zur Tagesordnung gefasst werden.
Davon ausgenommen ist lediglich ein Beschluss auf Einberufung einer außerordentlichen Generalversammlung, weil ein derartiger Antrag auch in der ordentlichen Generalversammlung gestellt werden kann.

(5)
In der Generalversammlung sind nur ordentliche Mitglieder (nach der Maßgabe in § 7 Abs 2), Ehrenmitglieder und Ehrenobmänner bzw. Ehrenpräsidenten, die bereits volljährig und bei Beginn der Generalversammlung anwesend sind, stimmberechtigt.

\subsection{§ 16
Qualifizierte Mehrheiten}

Nachfolgende Beschlüsse der Generalversammlung erfordern eine qualifizierte Mehrheit von zwei Dritteln der abgegebenen, gültigen Stimmen:

\begin{enumerate}[label=\alph*)]
\item
Beschluss über die Änderung der Vereinsstatuten;
\item
Beschluss über die freiwillige Auflösung des Vereins;
\item
Beschluss über die Enthebung des gesamten Vorstands, einzelner Vorstandsmitglieder, oder der Rechnungs- bzw. des Abschlussprüfers;
\item
Beschluss über die Aufnahme eines Antrags in die Tagesordnung einer Generalversammlung gem. § 13 Abs. 8 zweiter Satz;
\item
Beschlussfassung über alle Maßnahmen, die geeignet sind, die Befugnisse der Generalversammlung zu beeinträchtigen gem. §14 lit n).
\end{enumerate}

\subsection{§ 17
Der Vorstand}

(1)
Der Vorstand als Leitungsorgan des Vereins wird von der Generalversammlung gewählt.
Zu Vorstandsmitgliedern können nur natürliche Personen bestellt werden, die Vereinsmitglieder sein müssen.

(2)
Der Vorstand besteht aus mindestens 3 (drei) und höchstens 5 (fünf) Mitgliedern, und zwar namentlich aus dem Präsidenten, dem Vizepräsidenten und einem oder mehreren weiteren Vorstandsmitgliedern.

(3)
Scheidet der Präsident aus, so ist spätestens in der nächsten ordentlichen Generalversammlung ein neuer Vorstand mit neuer Funktionsperiode gemäß § 20 zu wählen.
In diesem Fall haben die verbleibenden Vorstandsmitglieder aus ihren Reihen einen Interimspräsidenten zu wählen, der bis zur Neuwahl des Präsidenten im Amt bleibt.

(4)
Falls die Höchstzahl an Mitgliedern des Vorstands nicht erreicht ist, kann der Vorstand jederzeit mittels Beschlusses ein wählbares Mitglied, welches auch Vereinsmitglied ist, für die restliche Funktionsperiode des Vorstandes kooptieren.

Eine Kooptierung in den Vorstand ist in der unmittelbar folgenden Generalversammlung mit einfacher Mehrheit der abgegebenen, gültigen Stimmen zu genehmigen.
Bis zu einer solchen Generalversammlung dürfen maximal 2 (zwei) Vorstände kooptiert werden, es sei denn, mangels Kooptierung würde die Mindestanzahl der Vorstandsmitglieder unterschritten werden.

(5)
Fällt der Vorstand ohne Selbstergänzung durch Kooptierung überhaupt oder auf unvorhersehbar lange Zeit aus, so ist jeder Rechnungsprüfer verpflichtet, unverzüglich eine außerordentliche Generalversammlung zum Zwecke der Neuwahl eines Vorstandes einzuberufen.
Sind auch die Rechnungsprüfer handlungsunfähig, hat jedes ordentliche Mitglied, das die Notsituation erkennt, unverzüglich die Bestellung eines Kurators beim zuständigen Gericht zu beantragen, der umgehend eine außerordentliche Generalversammlung einzuberufen hat.

(6)
Die Funktionsperiode des Vorstands beträgt 3 (drei) Jahre, das bedeutet, dass sie mit der dritten ordentlichen Generalversammlung nach der Wahl endet; eine Wiederwahl ist möglich.
Die Funktionsperiode währt auf jeden Fall bis zur Wahl eines neuen Vorstands.
Jede Funktion im Vorstand ist persönlich auszuüben.

(7)
Über Beschluss der Generalversammlung kann die Funktionsperiode des Vorstands einmalig bis 2 (zwei) Monate nach Abschluss des laufenden Geschäftsjahres verlängert werden.

(8)
Der Vorstand wird für seine Sitzungen vom Präsidenten, bei dessen Verhinderung vom Vizepräsidenten, schriftlich (per E-Mail) oder mündlich einberufen.
Ist auch der Vizepräsident auf unvorhersehbar lange Zeit verhindert, darf jedes sonstige Vorstandsmitglied den Vorstand einberufen.

Die hauptamtlich tätigen Manager können ohne Stimmrecht über Wunsch eines Vorstandsmitglieds an den Vorstandssitzungen teilnehmen, falls alle Vorstandsmitglieder damit einverstanden sind.

(9)
Der Vorstand ist beschlussfähig, wenn alle seine Mitglieder eingeladen wurden und mindestens die Hälfte der gewählten und kooptierten Vorstandsmitglieder anwesend ist (allenfalls auch Zuschaltung über Videokonferenz).

(10)
Um rechtsgültige Vorstandsbeschlüsse fassen zu können, bedarf es einerseits der Anwesenheit des Präsidenten oder Vizepräsidenten - ausgenommen bei Gefahr in Verzug oder wenn sowohl der Präsident als auch der Vizepräsident ihre Funktionen zurückgelegt haben bzw. dauerhaft handlungsunfähig sind - und andererseits der Anwesenheit von mindestens gleich viel oder mehr gewählter Vorstandsmitglieder als noch nicht bestätigter, kooptierter Vorstandsmitglieder.

(11)
Eine Beschlussfassung ist auch im Wege eines Umlaufbeschlusses möglich, wenn sämtliche Vorstandsmitglieder sich mit der Abstimmung im schriftlichen Wege einverstanden erklären.
Bei Abstimmung im schriftlichen Wege wird die erforderliche Mehrheit nicht anhand der Zahl der tatsächlich abgegebenen, sondern anhand der Zahl aller stimmberechtigter Vorstandsmitglieder berechnet.

(12)
Die Beschlüsse des Vorstands werden mit einfacher Stimmenmehrheit gefasst.
Bei Stimmengleichheit entscheidet die Stimme des Vorsitzenden.

(13)
Den Vorsitz in den Vorstandssitzungen führt der Präsident, bei dessen Verhinderung der Vizepräsident.
Ist auch dieser verhindert, obliegt der Vorsitz dem an Jahren ältesten, anwesenden Vorstandsmitglied oder jenem Vorstandsmitglied, das die übrigen Vorstandsmitglieder mehrheitlich dazu bestimmen.

(14)
Außer durch Tod und Ablauf der Funktionsperiode (Abs 6) erlischt die Funktion eines Vorstandsmitglieds durch Enthebung (Abs 15) durch die Generalversammlung und durch Rücktritt (Abs 16).

(15)
Die Generalversammlung kann jederzeit den gesamten Vorstand oder einzelne seiner Mitglieder entheben, wobei derartige Beschlüsse eine Mehrheit von zwei Dritteln der abgegebenen, gültigen Stimmen gem. § 16 lit. c) erfordern.
Die Enthebung tritt mit Bestellung des neuen Vorstands bzw. Vorstandsmitglieds in Kraft.

(16)
Die Vorstandsmitglieder können jederzeit schriftlich ihren Rücktritt erklären.
Die Rücktrittserklärung ist an den Vorstand, im Falle des Rücktritts des gesamten Vorstands an die Generalversammlung zu richten.
Wird die Zahl von drei (3) Vorstandsmitgliedern unterschritten, ist der Rücktritt eines Vorstandsmitglieds aufschiebend bedingt bis zu einer Kooptierung oder bis zum Ablauf jener Generalversammlung wirksam, die über eine Neuwahl entscheidet.

\subsection{§ 18
Aufgaben des Vorstands}

(1)
Der Vorstand ist jenes Organ des Vereins im Sinne des Vereinsgesetzes 2002, dem die Leitung des Vereines obliegt.

(2)
Dem Vorstand kommen alle Aufgaben zu, die nicht durch die Statuten einem anderen Vereinsorgan zugewiesen sind.

(3)
Zum Aufgabengebiet des Vorstands zählen insbesondere nachfolgende Angelegenheiten (demonstrative Aufzählung):

\begin{enumerate}[label=\alph*)]
\item
Einrichtung eines den Anforderungen des Vereins bzw. den Lizenzvorgaben der Österreichischen Fußball-Bundesliga entsprechenden Rechnungswesens mit laufender Aufzeichnung der Einnahmen und Ausgaben und der Führung eines Vermögensverzeichnisses als Mindesterfordernis;
\item
Erstellung des konsolidierten Jahresvoranschlages (Verein und Tochtergesellschaften), des Rechenschaftsberichtes und des konsolidierten Rechnungs- bzw. Jahresabschlusses (Verein und Tochtergesellschaften);
\item
Einberufung und Vorbereitung von ordentlichen und außerordentlichen Generalversammlungen;
\item
Information der Vereinsmitglieder über die Vereinstätigkeit, die Vereinsgebarung und den konsolidierten, geprüften Jahresabschluss;
\item
Verwaltung des Vereinsvermögens;
\item
Begründung und Auflösung der Dienstverhältnisse von Dienstnehmern des Vereins, inklusive allfälliger Geschäftsführer;
\item
Überwachung der Geschäftsführung;
\item
Gründung von Tochterfirmen oder Zweigvereinen;
\item
Aufnahme, Ausschluss und Streichung von Mitgliedern;
\item
Gründung und Konstituierung des Beirats gem. § 23 sowie Einberufung zu dessen konstituierenden Sitzung.
\end{enumerate}

(4)
Zur Regelung der inneren Organisation (Einsetzung von Ausschüssen, Geschäftsführung, Clubmanagement, u.a.m.) kann vom Leitungsorgan unter Berücksichtigung dieser Statuten eine Geschäftsordnung beschlossen werden.

\subsection{§ 19
Besondere Obliegenheiten einzelner Vorstandsmitglieder}

(1)
Der Präsident führt die laufenden Geschäfte des Vereins.
Die übrigen Vorstandsmitglieder unterstützen den Präsidenten bei der Führung der Vereinsgeschäfte.

(2)
Der Präsident vertritt den Verein nach außen.
Im Verhinderungsfall des Präsidenten wird der Verein vom Vizepräsidenten nach außen vertreten.
Sollte auch dieser verhindert sein, so ist jedes andere Vorstandsmitglied zur Vertretung des Vereins nach außen berufen.
Diese Vertretungsbefugnis kann vom Präsidenten oder Vizepräsidenten mit einer schriftlichen Vollmacht für bestimmte Geschäfte und Verfahren auch an Rechtsanwälte, Notare, Steuerberater und leitende Angestellte des Vereins delegiert werden.

(3)
Schriftliche Ausfertigungen des Vereins bedürfen zu ihrer Gültigkeit der Unterschriften des Präsidenten und eines weiteren Vorstandsmitglieds.
Diese können die Zeichnungsberechtigung für den Verein durch Vollmacht auch an einen leitenden Angestellten des Vereins übertragen.

(4)
Bei Gefahr in Verzug ist der Präsident berechtigt, auch in Angelegenheiten, die in den Wirkungsbereich der Generalversammlung oder des Vorstands fallen, eigenverantwortlich selbständige Anordnungen zu treffen.
Im Innenverhältnis bedürfen derartige Anordnungen jedoch der nachträglichen Genehmigung durch das zuständige Vereinsorgan.

(5)
Der Präsident führt den Vorsitz in den Generalversammlungen und in den Vorstandssitzungen.
Er hat die Einhaltung der Statuten zu überwachen und für die ordnungsgemäße Durchführung der gefassten Beschlüsse zu sorgen.

(6)
Der Vorstand kann eine interne Geschäftsordnung beschließen und dabei einzelnen Vorstandsmitgliedern bestimmte Aufgaben (z.B. Schriftführer, Finanzreferent, Rechtsreferent, usw.) zuweisen, für welche das jeweilige Vorstandsmitglied selbst verantwortlich zeichnet.

(7)
Rechtsgeschäfte zwischen Vorstandsmitgliedern und dem Verein bedürfen der Zustimmung eines anderen Vorstandsmitglieds.

\subsection{§ 20
Die Wahl des Vorstands}

(1)
Der Präsident und die sonstigen Mitglieder des Vorstands werden auf der Generalversammlung in getrennten Wahlgängen bestimmt.
Zunächst der Präsident im Verfahren der Einzelwahl, sodann ein von ihm vorgeschlagenes Team sonstiger Vorstandsmitglieder im Verfahren der Blockwahl.
Die Mitglieder des Vorstands gelten erst dann als bestellt, wenn sowohl Präsident als auch Team gewählt sind.
Die Wahlen sind in jedem Fall geheim durchzuführen.

(2)
\underline{Der Wahlvorschlag für die Generalversammlung wird wie folgt erstellt:}

\begin{enumerate}[label=\alph*)]
\item
Für die Funktion des Präsidenten ist beim Vorstand, organisatorisch vertreten durch das Clubmanagement, ein schriftlicher Wahlvorschlag einzubringen.
Jedes Mitglied ist berechtigt, ein ordentliches Mitglied zu nominieren.
Der Wahlvorschlag hat den vollständigen Namen, Meldeanschrift, Geburtsdatum und Geburtsort sowie die eigenhändige Unterschrift sowohl des Vorgeschlagenen als auch des Vorschlagenden zu enthalten.
Die Nominierungsfrist endet 2 (zwei) Wochen vor der Generalversammlung.
\item
Der nominierte Präsidentschaftskandidat hat sodann den Wahlvorschlag um sein Team der sonstigen Vorstandsmitglieder (Teamvorschlag) zu ergänzen.
Dabei hat der nominierte Präsidentschaftskandidat den zu wählenden Vizepräsidenten namentlich festzulegen.
Voraussetzung für die Nominierung als Vorstandsmitglied ist die ordentliche Mitgliedschaft.
Der Teamvorschlag hat den vollständigen Namen, Meldeanschrift, Geburtsdatum und Geburtsort sowie die eigenhändige Unterschrift der vorgeschlagenen Kandidaten für die Vorstandswahl zu enthalten.
Die Ergänzungsfrist endet 1 (eine) Woche vor der Generalversammlung.
\item
Das Clubmanagement hat die Mitglieder mit Ablauf der jeweiligen Fristen über die eingebrachten Vorschläge durch Veröffentlichung auf der offiziellen Vereinshomepage zu informieren.
Liegen mit Ablauf der jeweiligen Fristen keine Vorschläge vor, so kann der Vorstand eine Nachfrist setzen.
\item
Spätestens 1 (eine) Woche vor der Generalversammlung ist ein Vereinsabend durchzuführen.
Den durch Wahlvorschlag nominierten Kandidaten ist dabei Gelegenheit zur Präsentation einzuräumen.
Ab dem Vereinsabend kann der nominierte Präsidentschaftskandidat seinen Teamvorschlag bis zur Durchführung der Blockwahl auf der Generalversammlung jederzeit ändern.
\end{enumerate}

(3)
\underline{Das Wahlverfahren auf der Generalversammlung steht unter der Leitung des noch}
\underline{amtierenden Vorstands und gliedert sich wie folgt:}

\begin{enumerate}[label=\alph*)]
\item
Im ersten Wahlgang wird im Wege einer Einzelwahl der Präsident gewählt.
Gibt es mehrere Präsidentschaftskandidaten gilt jener Kandidat als gewählt, der die Mehrheit der abgegebenen, gültigen Stimmen erhält.
Stellt sich nur ein Kandidat der Wahl, so muss er die einfache Mehrheit der abgegebenen, gültigen Stimmen auf sich vereinigen können.
\item
Im zweiten Wahlgang wird im Wege einer Blockwahl das vom Präsidenten nominierte Vorstandsteam gewählt.
Die Wahl ist als Abstimmung mit den Wahlmöglichkeiten "`ja"' und "`nein"' durchzuführen.
Der Teamvorschlag gilt als angenommen, wenn er die einfache Mehrheit der abgegebenen, gültigen Stimmen erhält.
\item
Erhält der Teamvorschlag nicht die erforderliche Mehrheit, so kann der bereits gewählte Präsidentschaftskandidat diesen entweder
	\begin{itemize}
	\renewcommand\labelitemi{-}
	\item
	ändern und erneut zur Abstimmung gemäß lit b stellen, oder
	\item
	von seiner Wahl zurücktreten.
	\end{itemize}
\item
Das Recht zur Änderung des Teamvorschlags gemäß § 20 Abs 3 lit c kann nur einmal in Anspruch genommen werden.
Führt das Wahlverfahren auf der Generalversammlung zu keinem Ergebnis, so ist jeder Rechnungsprüfer verpflichtet, unverzüglich eine außerordentliche Generalversammlung zum Zwecke der Wahl eines Vorstandes einzuberufen.
Diese hat jedenfalls innerhalb einer Frist von 6 (sechs) Wochen stattzufinden.
Das Wahlverfahren startet erneut.
\item
Der amtierende Vorstand bleibt so lange im Amt, bis ein neuer Vorstand gewählt ist.
\end{enumerate}

\subsection{§ 21
Der Beirat}

(1)
Im ersten Jahr seiner Funktionsperiöde konstituiert der Vereinsvorstand gem.§ 23 den Beirat, welcher dem Vereinsvorstand beratend zur Seite steht.

(2)
Der Beirat besteht aus mindestens 3 (drei) und höchstens 7 (sieben) natürlichen Personen, die von den in § 25 lit. a-e genannten Gebietskörperschaften, Universitäten sowie der Fan-Initiative Innsbruck (ZVR:464544723) auf Einladung des Vorstands entsandt werden bzw. vom Beirat kooptiert werden.

(3)
Der Beirat wählt bei der konstituierenden Sitzung aus seiner Mitte einen Vorsitzenden und einen stellvertretenden Vorsitzenden für die Dauer der Funktionsperiode.
Außerdem kann sich der Beirat selbst eine mit den Bestimmungen der Satzung im Einklang stehende Geschäftsordnung geben.

(4)
Scheidet der Vorsitzende aus, so übernimmt der stellvertretende Vorsitzende seine Aufgaben und spätestens in der nächsten Sitzung wird die Funktion neu gewählt.
Scheidet der stellvertretende Vorsitzende aus, so ist spätestens in der nächsten Sitzung die Funktion neu zu wählen.

(5)
Falls die Höchstzahl an Mitgliedern des Beirats nicht erreicht ist, kann der Beirat jederzeit mittels Beschlusses weitere natürliche Personen, für die restliche Funktionsperiode des Beirats kooptieren.
Droht die Mindestzahl der Beiratsmitglieder unterschritten zu werden, ist eine Kooptierung verpflichtend in die Wege zu leiten.
Jede Kooptierung ist im Vorfeld mit dem Vorstand abzustimmen und von diesem mittels Beschluss zu genehmigen.

(6)
Die Funktionsperiode des Beirats beträgt 3 (drei) Jahre.
Sie beginnt und endet jedoch genau ein Jahr nach Beginn und Ende der Vorstandsperiode.
Die Funktionsperiode währt auf jeden Fall bis zur Konstituierung eines neuen Beirats.
Die Beiräte haben ihre Funktion persönlich auszuüben.

(7)
Der Beirat wird nach erfolgter Konstituierung zu seinen Sitzungen vom Vorsitzenden, bei dessen Verhinderung vom stellvertretenden Vorsitzenden schriftlich (per E-Mail) einberufen.

(8)
Der Beirat ist beschlussfähig, wenn alle seine Mitglieder eingeladen wurden und mindestens die Hälfte davon anwesend ist, in der sich auch der Vorsitzende oder der stellvertretende Vorsitzende befinden müssen (allenfalls auch Zuschaltung über Videokonferenz).

(9)
Die Beschlüsse des Beirats werden mit einfacher Stimmenmehrheit gefasst.
Bei Stimmengleichheit entscheidet die Stimme der Vorsitzführung.
Die Vorsitzführung obliegt dem gewählten Vorsitzenden des Beirats, bei dessen Verhinderung dem stellvertretenden Vorsitzenden.

(10)
Der Vorsitzende des Beirats verfügt über einen Sitz im Vorstand ohne Stimmrecht und ist zu sämtlichen Vorstandssitzungen zu laden.
Dessen Aufnahme in den Vorstand ohne Stimmrecht bedarf der Beschlussfassung durch den Vorstand.

(11)
Mindestens einmal pro Quartal soll eine Sitzung des Beirates stattfinden.
Zu diesen Sitzungen können sämtliche Vorstandsmitglieder, sowie Angestellte und Mitglieder des Vereins als Auskunftspersonen geladen werden.

(12)
Die Funktion eines Beirats erlischt durch Tod, Ablauf der Funktionsperiode (Abs 6) durch Neunominierung der entsendenden Institution (Abs 13) oder durch Rücktritt (Abs 14 ).

(13)
Die entsendenden Institutionen können jederzeit Neunominierung ihrer Beiräte vornehmen.
Die Neunominierung ist dem Vorstand, dem Vorsitzenden des Beirats sowie den davon betroffenen Beiräten zur Kenntnis zu bringen.

(14)
Die Mitglieder des Beirats können jederzeit schriftlich ihren Rücktritt erklären.
Die Rücktrittserklärung ist an den Vorstand, den Vorsitzenden des Beirats sowie die entsendende Institution zu richten.

(15)
Die Tätigkeit als Beirat ist von Vereinsseite grundsätzlich als Ehrenamt zu betrachten.
Einzig dem Vorsitzenden des Beirats und seinem Stellvertreter kann nach Maßgabe des tatsächlichen Zeitaufwandes eine Aufwandsentschädigung gewährt werden.

\subsection{§ 22
Aufgaben des Beirates}

(1)
Der Beirat ist jenes Organ des Vereins das dem Vorstand bei der Leitung des Vereins beratend zur Seite steht.

(2)
Kernaufgabe des Beirates ist es im Rahmen der Beiratssitzungen Beschlüsse zu fassen, die dem Vorstand Handlungsempfehlungen oder Unterlassungsempfehlungen aussprechen.
Diese Empfehlungen erstrecken sich auf sämtliche Aufgaben des Vorstandes und dürfen den Bestimmungen der Statuten nicht zuwiderlaufen.

(3)
Empfehlungen des Beirates sind innerhalb von 5 (fünf) Tagen nach erfolgtem Beschluss dem Vorstand schriftlich zur Kenntnis zu bringen.

(4)
Der Vereinsvorstand hat diese Empfehlungen in der nächstfolgenden Vorstandssitzung zu erörtern und dem Beirat schriftlich mitzuteilen, ob, inwieweit und in welcher Form der Verein diese Empfehlungen umzusetzen gedenkt.
Das Verwerfen von Empfehlungen ist zu begründen.

\subsection{§ 23
Konstituierung des Beirates}

(1)
Der Vorstand hat im ersten Jahr seiner Tätigkeit (siehe§ 21 Abs.1)

\begin{enumerate}[label=\alph*)]
\item
das Land Tirol über das Büro des für Sport zuständigen Landesrates schriftlich einzuladen, \textbf{einen (1) Beirat} zu entsenden;
\item
die Stadt Innsbruck über das Büro der zuständigen Stadträtin für Sport schriftlich einzuladen, \textbf{einen (1) Beirat} zu entsenden;
\item
die Universität Innsbruck (LFU) über deren Abteilung für Öffentlichkeitsarbeit schriftlich einzuladen, \textbf{einen (1) Beirat} zu entsenden;
\item
die Medizinische Universität Innsbruck (MUI) über deren Abteilung für Öffentlichkeitsarbeit schriftlich einzuladen, \textbf{einen (1) Beirat} zu entsenden;
\item
die Fan-Initiative Innsbruck e.V. über deren Vorstand schriftlich einzuladen, \textbf{zwei (2) Beiräte} zu entsenden.
\end{enumerate}

(2)
Nach erfolgter Entsendung gem. Abs. 1 hat der Vorstand die nominierten Beiräte zur konstituierenden Sitzung einzuladen.
Diese hat jedenfalls so rechtzeitig zu erfolgen, dass der Beirat zu Beginn seiner Funktionsperiode (§ 21 Abs.6) konstituiert ist.

(3)
Die Konstituierung des Beirats kann nur erfolgen, wenn die erforderliche Mindestanzahl an Mitgliedern (§ 21 Abs. 2) nominiert wurde und er bei der konstituierenden Sitzung beschlussfähig(§ 21 Abs. 8) ist.

(4)
Bis die Wahl eines Vorsitzenden erfolgt ist, leitet ein Vorstandsmitglied die konstituierende Sitzung.
Nach erfolgter Wahl führt der Vorsitzende die Sitzung weiter.

\subsection{§ 24
Das Schiedsgericht}

(1)
Zur Schlichtung von allen aus dem Vereinsverhältnis entstehenden Streitigkeiten ist das vereinsinterne Schiedsgericht berufen.
Es ist eine "`Schlichtungseinrichtung"' im Sinne des Vereinsgesetzes 2002 und kein Schiedsgericht nach den §§ 577 ff ZPO.

(2)
Das Schiedsgericht setzt sich aus 5 (fünf) ordentlichen Vereinsmitgliedern zusammen, die keinem Organ - mit Ausnahme der Generalversammlung - angehören dürfen, dessen Tätigkeit Gegenstand der Streitigkeit ist.
Es wird derart gebildet, dass jeder Streitteil dem Vorstand 2 (zwei) ordentliche Mitglieder als Schiedsrichter schriftlich namhaft macht, wobei der Vorstand, ist er selbst bzw. der Verein der andere Streitteil, innerhalb von 14 (vierzehn) Tagen die beiden weiteren Mitglieder des Schiedsgerichtes namhaft zu machen hat; ist ein anderes Vereinsmitglied vom Streit betroffen, so fordert der Vorstand dieses Mitglied auf, innerhalb von 14 (vierzehn) Tagen ab Zustellung der Aufforderung zwei weitere Mitglieder des Schiedsgerichtes namhaft zu machen.

(3)
Diese beiden Schiedsrichter wählen mit einfacher Stimmenmehrheit eine fünfte Person zum Vorsitzenden des Schiedsgerichtes.
Können sie sich nicht binnen 7 (sieben) Tagen einigen, so entscheidet das Los.
Die Schiedsrichter sind verpflichtet, sich an der Auslosung zu beteiligen.
Verhindert ein nominierter Schiedsrichter das Zustandekommen oder Arbeiten des Schiedsgerichtes, so ist dies dem Mitglied zuzurechnen, welches ihn nominiert hat.
Dieses Mitglied ist vom Vorstand sodann aufzufordern, binnen einer angemessenen Frist für Ersatz zu sorgen.

(4)
Das Schiedsgericht versucht zunächst eine Schlichtung.
Ist eine solche nicht möglich, ist es zur Entscheidung der Streitsache befugt.
Die Streitteile können sich rechtsanwaltlich vertreten lassen, ein Kostenzuspruch findet jedoch nicht statt.

(5)
Die Einberufung des Schiedsgerichts hat unverzüglich durch den Vorsitzenden zu erfolgen.
Das Schiedsgericht entscheidet, ohne an bestimmte Vorschriften gebunden zu sein, nach bestem Wissen und Gewissen.
Es trifft seine Entscheidungen bei Anwesenheit aller seiner Mitglieder mit einfacher Stimmenmehrheit.
Das Schiedsgericht hat auf eine zügige Bearbeitung und einen möglichst schnellen Abschluss des Schiedsverfahrens zu achten.
Das Schiedsgericht kann eine mündliche Verhandlung mit Beteiligung der Streitparteien ansetzen, sofern es dies für zweckdienlich erachtet.
Der Vorsitzende des Schiedsgerichtes ist für die Ausfertigung der Entscheidung verantwortlich, die jedenfalls eine Begründung zu enthalten hat.
Die Entscheidungen sind vereinsintern endgültig.

(6)
Vereinsstreitigkeiten, die keine Rechtsstreitigkeiten sind, entscheidet das Schiedsgericht endgültig.

(7)
Nennt der Antragsgegner binnen einer Frist von 14 (vierzehn) Tagen nach Nennung der Schiedsrichter durch den Antragsteller keine zwei Schiedsrichter oder nennt er binnen angemessener Frist kein Ersatzmitglied, so gilt der Streitgegenstand als anerkannt.

(8)
Mitglieder, die sich in einer Streitigkeit aus dem Vereinsverhältnis nicht dem Schiedsgericht unterwerfen oder dessen Entscheidungen nicht anerkennen, können aus dem Verein ausgeschlossen werden.

\subsection{§ 25
Die Rechnungsprüfer bzw. Abschlussprüfer}

(1)
Zwei Rechnungsprüfer, die fachkundig sein müssen, werden von der Generalversammlung für die Funktionsdauer des Vorstandes gewählt.
Eine Wiederwahl ist möglich.
Die Rechnungsprüfer dürfen keinem Organ - mit Ausnahme der Generalversammlung - angehören, dessen Tätigkeit Gegenstand der Prüfung ist.

(2)
Die Rechnungsprüfer werden im Wege einer Einzelwahl gewählt.
Sie gelten als gewählt, wenn sie jeweils die Mehrheit der abgegebenen, gültigen Stimmen erhalten.

(3)
Den Rechnungsprüfern obliegen die laufende Geschäftskontrolle sowie die Prüfung der Finanzgebarung des Vereins im Hinblick auf die Ordnungsmäßigkeit der Rechnungslegung und die statutengemäße Verwendung der Mittel.
Der Vorstand hat den Rechnungsprüfern die erforderlichen Unterlagen vorzulegen und die erforderlichen Auskünfte zu erteilen.
Die Rechnungsprüfer haben dem Vorstand über das Ergebnis der Prüfung zu berichten, der dieses Ergebnis in weiterer Folge der Generalversammlung vorlegt.

(4)
Rechtsgeschäfte zwischen Rechnungsprüfern und Verein bedürfen der Genehmigung durch die Generalversammlung.
Im Übrigen gelten für die Rechnungsprüfer die Bestimmungen des § 17 Abs 14 bis 16 sinngemäß.

(5)
Erfüllt der Verein die Voraussetzung des§ 22 Abs 2 Vereinsgesetz 2002, so gelten die Bestimmungen über die Rechnungsprüfer sinngemäß für den Abschlussprüfer.

(6)
Besteht gemäß § 22 Abs. 2 Vereinsgesetz die gesetzliche Verpflichtung zur Abschlussprüfung, so übernimmt der Abschlussprüfer zwingend die Aufgaben der Rechnungsprüfer.
In diesem Fall sind keine Rechnungsprüfer zu wählen.
Dies gilt auch dann, wenn eine Abschlussprüfung freiwillig im Umfang des § 22 Abs. 2 Vereinsgesetz beauftragt wird.

\section{IV. SCHLUSSBESTIMMUNGEN}

\subsection{§ 26
Freiwillige Auflösung des Vereins}

(1)
Die freiwillige Auflösung des Vereines kann nur in einer zu diesem Zweck einberufenen außerordentlichen Generalversammlung und nur mit einer Mehrheit von zwei Dritteln der abgegebenen, gültigen Stimmen beschlossen werden.

(2)
Diese Generalversammlung hat auch - sofern Vereinsvermögen vorhanden ist - über die Abwicklung zu beschließen.
Insbesondere hat sie einen Abwickler zu berufen und Beschluss darüber zu fassen, wem dieser das nach Abdeckung der Passiva allenfalls verbleibende Vermögen zu übertragen hat.

(3)
Der Verein hat der Vereinsbehörde das Datum der freiwilligen Auflösung und, falls Vermögen vorhanden ist, das Erfordernis der Abwicklung sowie den Namen, das Geburtsdatum, den Geburtsort und die für Zustellungen maßgebliche Anschrift sowie den Beginn der Vertretungsbefugnis eines allenfalls bestellten Abwicklers binnen vier Wochen nach der Auflösung mitzuteilen.

(4)
Bei Auflösung des Vereins oder bei Wegfall des bisherigen begünstigten Vereinszwecks darf das allenfalls vorhandene Vereinsvermögen in keiner wie immer gearteten Form den Vereinsmitgliedern zugutekommen, sondern ist für sportliche gemeinnützige Zwecke im Sinne der§§ 34ff BAO zu verwenden.

\subsection{§ 27
Geschlechtsspezifische Bezeichnungen}

Alle Personenbezeichnungen, die in diesen Vereinsstatuten in der männlichen Form verwendet werden, gelten sinngemäß auch für die weibliche Form.
Werden Funktionen von Frauen ausgeübt, so wird die weibliche Form der Bezeichnung verwendet.

\subsection{§ 28
Inkrafttreten}

Die vorliegenden Statuten treten nach Beschlussfassung durch die Generalversammlung mit Beginn des kommenden Geschäftsjahres am 01.07.2022 in Kraft.
Mit Inkrafttreten werden alle früheren Vereinsstatuten aufgehoben.

\end{document}
