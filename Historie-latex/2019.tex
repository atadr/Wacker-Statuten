\documentclass[10pt,a4paper]{article}
\usepackage[a4paper, total={6.5in, 9.5in}]{geometry}
\usepackage[ngerman]{babel}
\usepackage{hyphenat}
\usepackage{lmodern}
\usepackage{parskip}
\usepackage{enumitem}
\usepackage{sectsty}
\sectionfont{\centering}

\author{FC Wacker Innsbruck}
\title{Statuten des FC Wacker Innsbruck}
\begin{document}
\setlength{\parskip}{25pt}
\setcounter{secnumdepth}{0}

\section{STATUTEN DES VEREINS FC WACKER INNSBRUCK\\(ZVR-Zahl: 180936836)}

\textbf{Fassung gemäß Beschluss der Generalversammlung vom 27.01.2019}

\section{Inhalt:}

\begin{itemize}
\item
PRÄAMBEL

\item
ERSTES HAUPTSTÜCK: Die äußere und innere Identität des FC Wacker Innsbruck

\item
ZWEITES HAUPTSTÜCK: Die institutionelle Architektur des FC Wacker Innsbruck - die Organe und ihre Wirkungsweise

\item
DRITTES HAUPTSTÜCK: Das demokratische Leben des FC Wacker Innsbruck

\item
VIERTES HAUPTSTÜCK: Schlussbestimmungen
\end{itemize}

\section{PRÄAMBEL}

Geleitet von dem Willen der Mitglieder, ihre Zukunft gemeinsam zu gestalten, begründen diese Statuten den rechtlichen und das Markengrundbuch den Werte-Rahmen des FC Wacker Innsbruck.
Unser Verein steht allen offen, die seine Werte und Grundsätze achten und im positiven Sinne nach Außen tragen.

In Übereinstimmung mit allen verantwortlichen Personen, denen das Wohlergehen und die nachhaltige Weiterentwicklung des offenen, demokratischen Mitgliedervereines FC Wacker Innsbruck am Herzen liegen, wurde das Markengrundbuch erarbeitet und niedergeschrieben.
Das Markengrundbuch erfährt nunmehr jenen vorrangigen Stellenwert, der es allen Beteiligten ermöglicht, nach den beschriebenen Inhalten und Werten zu leben, zu handeln und diese auch von anderen einzufordern.
Damit bringt der FC Wacker Innsbruck seine Einzigartigkeit zum Ausdruck.

Auf Grundlage dieser Werte begreift der Verein den Fußballsport und sich selbst als Kulturgut und entwickelt daraus seine spezifische Vereinskultur (Markengrundbuch - Version 2009 in seiner jeweils geltenden Fassung).

Im Verständnis und im Bewusstsein, dass diese Statuten die rechtliche Verkörperung unserer Wertekultur bilden, kommen wir, die Mitglieder, wie folgt überein:

\clearpage
\setlength{\parskip}{10pt}
\upshape

\section{ERSTES HAUPTSTÜCK: Die äußere und innere Identität des FC Wacker Innsbruck}

\subsection{§ 1
Name, Sitz und Tätigkeitsbereich}

(1)
Der Verein führt den Namen "`FC Wacker Innsbruck"'.

(2)
Er hat seinen Sitz in Innsbruck und erstreckt seine Tätigkeit auf das gesamte Gebiet der Europäischen Union (EU).
Die Sportausübung ist auch außerhalb der EU möglich.

(3)
Die Errichtung von Zweigvereinen ist nicht beabsichtigt.
Der Verein kann jedoch Beteiligungen an juristischen Personen begründen, diese verwalten und finanzieren sowie solche Beteiligungen wieder aufgeben; § 11 Abs 1 Z 8 bleibt hievon unberührt.

\subsection{§ 2
Vereinsfarben und Vereinswappen}

(1)
Die Vereinsfarben sind "`Schwarz-Grün"'.

(2)
Das Vereinswappen visualisiert den Ursprung, die Identität und die Zukunft des FC Wacker Innsbruck.
Das Ursprungswappen ist Hauptbestandteil des Vereinswappens und eingebettet in das schwarz-grüne Selbstverständnis.

\subsection{§ 3
Zweck}

(1)
Der Verein, dessen Tätigkeit nicht auf Gewinn gerichtet ist, bezweckt die Pflege und Förderung des Körpersportes jeder Art, insbesondere die Ausübung des Nachwuchs-, Amateur- und Spitzenfußballs unter Ausschluss jeder politischen Tendenz.
Das hierbei vom Verein vorgehaltene Mannschaftsangebot umfasst die nach den Bestimmungen des österreichischen Fußballbundes (ÖFB) und der österreichischen Fußball-Bundesliga für die Teilnahme an den betroffenen Bewerben verpflichtende Anzahl, mindestens jedoch jeweils zwei Mannschaften für den Herren-, Damen- und Nachwuchsbereich.

(2)
Der Verein wird auf gemeinnütziger Basis von einem gewählten Vorstand ehrenamtlich geführt.

(3)
Mitarbeiter oder Mitglieder von Organen von Unternehmen, die zu mehreren Vereinen oder Tochtergesellschaften der Lizenzligen bzw. Muttervereinen in vertraglicher Beziehungen stehen, dürfen nicht Mitglied in Kontroll-, Geschäftsführungs- und Vertretungsorganen sein, es sei denn, ein wirtschaftliches Interesse dieser Unternehmen an sportlichen Erfolgen mehrerer Vereine oder Tochtergesellschaften der Lizenzligen oder Muttervereinen ist auszuschließen.

(4)
Mitglieder von Kontroll-, Geschäftsführungs- und Vertretungsorganen anderer Vereine oder anderen Tochtergesellschaften der Lizenzligen oder anderer Muttervereine dürfen keine Funktionen in Organen des Vereins übernehmen.

(5)
Auch die sportliche Ausbildung sowie eine institutionalisierte Weiterbildung insbesondere zur fußball- und vereinsspezifischen Kompetenzbildung der Mitglieder ist Ziel des Vereines.

\subsection{§ 4
Mittel zur Erreichung des Vereinszweckes}

(1)
Der Vereinszweck soll durch die in den Absätzen 2 und 3 angeführten ideellen und materiellen Mittel erreicht werden.

(2)
Als ideelle Mittel dienen

\begin{enumerate}[label=\alph*)]
\item
die Teilnahme an verschiedenen Fußball-Meisterschaften, sowie die Pflege des Fußballsportes auf allen Gebieten des Breiten-, Spitzen- und Gesundheitssportes;
\item
sportliche Veranstaltungen;
\item
regelmäßige Vereinsabende in angemessener Anzahl (mindestens einmal im Quartal) und sonstige Maßnahmen der Mitglieder- und sozialer Fanbetreuung;
\item
Vorträge, Mitgliederkompetenzbildung (§ 3 Abs 5) und Versammlungen, Symposien und Diskussionsforen;
\item
Herausgabe einer Stadionzeitung und sonstiger Publikationen;
\item
Homepage;
\item
Aufbau eines wissenschaftlichen Think-Tanks;
\item
Aufbau einer Fachbibliothek, eines Museums sowie eines Dokumentationszentrums;
\item
Aufbau eines Trainingszentrums und einer Akademie;
\end{enumerate}

(3)
Die erforderlichen materiellen Mittel des Vereins sollen aufgebracht werden durch

\begin{enumerate}[label=\alph*)]
\item
Mitgliedsbeiträge
\item
Erträgnisse aus Veranstaltungen des Vereins
\item
Sponsorenbeiträge
\item
Erträgnisse sonstiger finanzieller Verwertung (z.B. Sportanlagen, Lizenzierung von Namensrechten und sonstigen Immaterialgüterrechten an mit dem Verein verbundene juristische Personen, Weitergabe von Teilnahmerechten am Spielbetrieb von Fußballligen an ebensolche juristische Personen etc.)
\item
Erträgnisse aus Beteiligung an Kapitalgesellschaften
\item
Erträgnisse aus Publikationen des Vereins und Förderungen
\item
Spenden, Vermächtnisse und sonstige Zuwendungen
\end{enumerate}

(4)
Die Vereinsgelder dürfen nur zur Deckung von Ausgaben für Vereinszwecke verwendet werden.
Die Mittel des Vereines dürfen ausschließlich für die begünstigten Zwecke verwendet werden.
Die Mitglieder des Vereines dürfen keine Gewinnanteile und in ihrer Eigenschaft als Mitglieder auch keine sonstigen Zuwendungen aus Mitteln des Vereines erhalten.
Es darf keine Person durch Verwaltungsausgaben, die dem Zweck des Vereines fremd sind oder durch unverhältnismäßig hohe Vergütungen begünstigt werden.

(5)
Der Verein ist berechtigt, die zur sportlichen Verwertung seiner Anlagen allenfalls erforderlichen Gewerbe nach vorheriger Einholung der behördlichen Genehmigung zur Ausübung der Gewerbeberechtigung anzumelden und auszuüben.
Die Einnahmen dürfen nur im Sinne des Vereinszweckes verwendet werden.

\subsection{§ 5
Arten der Mitgliedschaft}

(1)
Der demokratische Mitgliederverein "`FC Wacker Innsbruck"' gliedert sich in

\begin{enumerate}[label=\alph*)]
\item
ordentliche Mitglieder
\item
fördernde Mitglieder
\item
Ehrenmitglieder
\end{enumerate}

(2)
Arten der Mitgliedschaft:

\begin{enumerate}[label=\alph*)]
\item
Ordentliche Mitglieder sind jene, die aufgrund einer Beitrittserklärung einen jährlichen Mitgliedsbeitrag leisten.
Dieser wird von der Generalversammlung festgelegt und richtet sich nach Alter und Lebensabschnitt.
\item
Fördernde Mitglieder sind jene, die die Zwecke des Vereins durch Zahlung eines jährlichen Beitrags in der Höhe, die ebenfalls von der Generalversammlung festgelegt wird, oder durch sonstige Zuwendungen jeglicher Art fördern.
\item
Ehrenmitglieder sind Personen, die hierzu wegen besonderer Verdienste um den Verein ernannt werden.
\end{enumerate}

\subsection{§ 6
Erwerb der Mitgliedschaft}

(1)
Mitglieder des Vereins können nur physische Personen werden.

(2)
Über die Aufnahme der Mitglieder entscheidet der Vorstand mit einfacher Mehrheit.
Die Ablehnung einer Aufnahme kann ohne Angabe von Gründen erfolgen.

(3)
Die Ernennung zum Ehrenmitglied erfolgt auf Antrag des Vorstandes durch die Generalversammlung.

(4)
Ist der Mitgliedswerber noch nicht volljährig, so bedarf die Beitrittserklärung der Zustimmung seines gesetzlichen Vertreters.

\subsection{§ 7
Beendigung der Mitgliedschaft}

(1)
Die Mitgliedschaft erlischt durch Tod, durch freiwilligen Austritt, durch Streichung und durch Ausschluss.

(2)
Der freiwillige Austritt kann nur zum 30. Juni jedes Jahres erfolgen, die Austrittserklärung muss bis längstens 1. Juni desselben Jahres dem Vorstand des Vereines zugegangen sein.
Der Austritt entbindet nicht von der Erfüllung der bis zum Austrittszeitpunkt entstandenen Verbindlichkeiten gegenüber dem Verein.

(3)
Die Streichung eines Mitgliedes kann der Vorstand vornehmen, wenn dieses bis spätestens 1.12. des laufenden Geschäftsjahres mit der Zahlung des Mitgliedsbeitrages trotz Mahnung im Rückstand ist.
Gegen diese Streichung steht den Mitgliedern innerhalb eines Monats das Recht der Berufung an die nächste Generalversammlung zu.
Die Streichung entbindet ebenfalls nicht von der Erfüllung der bis zum Streichungszeitpunkt entstandenen Verbindlichkeiten gegenüber dem Verein.

(4)
Der Ausschluss eines Mitgliedes aus dem Verein kann vom Vorstand wegen grober Verletzung der Mitgliedspflichten und wegen unehrenhaften und vereinsschädigenden Verhaltens verfügt werden.
Gegen den Ausschluss ist die Berufung an die Generalversammlung binnen einem Monat zulässig, bis zu deren Entscheidung die Mitgliedsrechte ruhen.

(5)
Die Aberkennung der Ehrenmitgliedschaft kann aus den in Abs 4 genannten Gründen von der Generalversammlung auf Antrag des Vorstandes beschlossen werden.

\subsection{§ 8
Rechte und Pflichten der Mitglieder}

(1)
Allen Mitgliedern steht die freie und tätige Mitwirkung am Vereinsleben im Rahmen der Vereinsstatuten offen.
Sie üben ihre Teilhabe auf der Grundlage von Offenheit, Respekt und Verantwortung aus.

\begin{enumerate}[label=\alph*)]
\item
Nur ordentliche Mitglieder, deren Mitgliedschaft seit mindestens einem Jahr ununterbrochen besteht, haben das aktive Wahlrecht sowie das Stimmrecht in der Generalversammlung.
Das Erfordernis der ununterbrochenen einjährigen Mitgliedschaft ist auch dann erfüllt, wenn diese entweder zur Gänze oder in Teilen fördernder Natur gewesen ist.
\item
Nur ordentliche Mitglieder haben das passive Wahlrecht.
\item
Für die Erlangung des aktiven und des passiven Wahlrechts sowie des Stimmrechts in der Generalversammlung ist die Volljährigkeit Voraussetzung.
Fördernde Mitglieder können ihre Mitgliedschaft jederzeit in eine ordentliche umwandeln.
\end{enumerate}

(2)
Alle Mitglieder haben regelmäßige Mitgliedsbeiträge zu leisten.
Diese werden von der Generalversammlung festgesetzt und sind im Vorhinein zu entrichten, jedoch bis spätestens 30.6. eines jeden Jahres.

(3)
Alle Mitglieder sind verpflichtet, die Interessen des Vereins nach besten Kräften zu fördern, seine Werte zu achten und alles zu unterlassen, wodurch Ansehen und Zweck des Vereins leiden könnten.
Sie haben die Statuten und ihre Durchführungsbestimmungen sowie die Beschlüsse der Vereinsorgane genau zu beachten.

(4)
Die Mitglieder verpflichten sich, jeglichem rassistischen Verhalten im Stadion und im Klub entschieden entgegenzutreten, sowie zur Förderung von ethnischen Minderheiten und Migranten im Verein und zur Zusammenarbeit mit allen Organisationen, welche dem Problem des Rassismus im Fußball entgegenwirken.
Insbesondere Aktivitäten im Rahmen des Netzwerkes "`FARE - Football Against Racism in Europe"' und der österreichischen Kampagne "`FairPlay - Viele Farben. Ein Spiel"' sollen unterstützt werden.
Außerdem anerkennen die Mitglieder die Grundsätze des Zehn Punkte Plans der UEFA zur Bekämpfung des Rassismus im Fußball.

\section{ZWEITES HAUPTSTÜCK\\Die institutionelle Architektur des FC Wacker Innsbruck\\– die Organe und ihre Wirkungsweise}

\subsection{§ 9
Organe des Vereines}
(1)
Die Organe des Vereines sind:

\begin{enumerate}[label=\alph*)]
\item
die Generalversammlung
\item
der Vorstand
\item
die Rechnungsprüfer
\item
das Schiedsgericht
\end{enumerate}

(2)
Jedes Organ handelt nach Maßgabe der ihm in diesen Statuten zugewiesenen Zuständigkeiten und Wirkungsweisen.
Die Organe arbeiten loyal und koordiniert zusammen.

\subsection{§ 10
Die Generalversammlung}

(1)
Die ordentliche Generalversammlung findet einmal jährlich innerhalb von 8 Monaten nach Beendigung des Geschäftsjahres statt.
Sie wird durch den Vorstand einberufen.

(2)
Eine außerordentliche Generalversammlung ist innerhalb von 21 Tagen vom Vorstand einzuberufen:

\begin{enumerate}[label=\alph*)]
\item
auf Beschluss des Vorstandes
\item
auf Antrag der Rechnungsprüfer
\item
auf Antrag von mindestens einem Zehntel der stimmberechtigten Mitglieder;
sofern ein Antrag von stimmberechtigten Mitgliedern dieses Quorum nicht erreicht, hat der Vorstand in einer Vorstandssitzung beschlussmäßig zu entscheiden, ob er einen solchen Antrag unterstützt und somit von seinem Einberufungsrecht (Abs. (2) lit. a)) Gebrauch macht.
\end{enumerate}

(3)
In den Fällen von §§ 12 Abs 2 lit d und 20 Abs 3 lit d obliegt die Einberufung den Rechnungsprüfern.
Der Antrag auf eine außerordentliche Generalversammlung hat gleichzeitig jene(n) Antrag/Anträge zu enthalten, die bei der außerordentlichen Generalversammlung auf die Tagesordnung gebracht werden sollen.
Die Antragspunkte müssen der Beschlussfassung der Generalversammlung unterliegen, andernfalls gilt der Antrag als nicht gestellt.
Ein Antrag ohne Angabe von Gründen ist nichtig.

(4)
Verstreicht trotz ordnungsgemäßem Antrag auf Einberufung einer außerordentlichen Generalversammlung die Frist, ohne dass der Vorstand entsprechend tätig wird, so sind diejenigen, die den Antrag ordnungsgemäß eingebracht haben, berechtigt, unter Wahrung der vorgenannten Formen und Fristen die Generalversammlung selbst auf Kosten des Vereins einzuberufen.

(5)
Die außerordentliche Generalversammlung hat innerhalb von 42 Tagen ab Antrag oder Beschluss stattzufinden.
Der Tag der ordentlichen bzw. außerordentlichen Generalversammlung ist allen Mitgliedern mindestens 21 Tage vorher schriftlich - bei Hinterlegung der e-mail Adresse per e-mail ansonsten durch einen Post-Brief – sowie durch eine Einschaltung in den örtlichen Medien bekannt zu geben.
Die Anberaumung der Generalversammlung hat unter Angabe der vorläufigen Tagesordnung zu erfolgen.

(6)
Die Mitglieder haben die Möglichkeit, bis 14 Tage vor der Generalversammlung, Anträge an die Generalversammlung zu stellen.
Die Mitglieder müssen 10 Tage vor der ordentlichen Generalversammlung schriftlich über Tagesordnungspunkte und geplante Statutenänderungen informiert werden.

(7)
Anträge, die erst in der Versammlung gestellt werden, werden nur behandelt, wenn der Vorstand der Behandlung zustimmt oder die Versammlung die Behandlung mit Dreiviertelmehrheit beschließt.

(8)
Die Generalversammlung ist unabhängig von der Anzahl der anwesenden Mitglieder beschlussfähig.

(9)
Bei der Generalversammlung sind nur die ordentlichen Mitglieder stimmberechtigt, wobei jedes Mitglied nur eine Stimme hat.
Die Teilnahme an der Generalversammlung steht allen Vereinsmitgliedern zu.
Die Übertragung des Stimmrechts auf ein anderes Mitglied im Wege einer schriftlichen Bevollmächtigung ist nicht zulässig.

(10)
Die Wahlen und Beschlussfassungen erfolgen in der Regel mit einfacher Stimmenmehrheit.
Bei Stimmengleichheit gilt der Antrag als abgelehnt.

(11)
Den Vorsitz in der Generalversammlung führt der Präsident, in dessen Verhinderung ein anderes Mitglied des Vorstandes.

(12)
Generalversammlungen haben am Sitz des Vereins stattzufinden.

\subsection{§ 11
Aufgabenkreis der Generalversammlung}

Der Generalversammlung, die das oberste Beschlussorgan des Vereins ist, sind folgende Aufgaben vorbehalten:
\begin{enumerate}
\item
Entgegennahme und Genehmigung des Rechenschaftsberichtes des Vorstandes und des Jahresabschlusses unter Einbindung der Rechnungsprüfer;
\item
Beschlussfassung über Voranschlag;
\item
Wahl und Enthebung der Mitglieder des Vorstandes und der Rechnungsprüfer, wobei letztere keinem weiteren Organ als der Generalversammlung angehören dürfen.
\item
Entlastung des Vorstandes;
\item
Beschlussfassung über die Höhe der Mitgliedsbeiträge;
\item
Entscheidungen über Berufungen gegen Ausschlüsse von der Mitgliedschaft;
\item
Beschlussfassung über Änderungen der Statuten und die freiwillige Auflösung des Vereins.
Eine Statutenänderung ist in einer Generalversammlung mit einer Mehrheit von zwei Drittel der gültigen Stimmen zu beschließen.
\item
Beschlussfassung über alle Maßnahmen, die geeignet sind, die Befugnisse der Generalversammlung zu beeinträchtigen – sohin insbesondere die Überführung der Profiabteilung in eine eigene Rechtsform, die Lizenzübertragung zur Teilnahme an österreichischen Ligen oder die Veräußerung von Anteilen (insbesondere Verringerung der Stammeinlage oder eines sonstigen nominellen Kapitalanteiles) an jener juristischen Person, auf die die Profiabteilung übertragen wird bzw. wurde, sowie die Übertragung von Stimmrechten als Gesellschafter dieser juristische Person.
Die Beschlussfassung darüber erfordert die Mehrheit von zwei Drittel der gültigen Stimmen in einer Generalversammlung.
\item
Beratung und Beschlussfassung über sonstige auf der Tagesordnung stehende Fragen;
\item
Genehmigung von Rechtsgeschäften zwischen Rechnungsprüfern und dem Verein;
\item
Verleihung und Aberkennung der Ehrenmitgliedschaft;
\end{enumerate}

\subsection{§ 12
Der Vorstand}

(1)
Der Vorstand wird durch Wahl gemäß § 20 bestellt und besteht aus mindestens 3 Mitgliedern, namentlich dem Präsidenten, dem Vizepräsidenten und einem oder mehreren weiteren Vorstandsmitgliedern.
Die Aufgaben eines Schriftführers, Finanzreferenten und Rechtsreferenten sind jeweils durch die einzelnen Vorstandsmitglieder auszuführen; der Vorstand weist diese Funktionen unter Bedachtnahme auf die Kompetenz der Vorstandsmitglieder durch Beschluss den einzelnen Vorstandsmitgliedern zu.

(2)
Der Vorstand hat das Recht, durch Wahl für die restliche Funktionsperiode an Stelle von ausscheidenden Vorstandsmitgliedern oder auch zusätzlich weitere wählbare natürliche Personen, die auch Vereinsmitglieder sind, in den Vorstand zu kooptieren; hierbei ist folgendes zu beachten:

\begin{enumerate}[label=\alph*)]
\item
Scheidet der Präsident aus, so ist in der nächsten Generalversammlung ein neuer Vorstand mit neuer Funktionsperiode gemäß § 20 gewählt zu werden;
bis zur Neuwahl übernimmt der Vizepräsident die Funktion des Präsidenten.
\item
Scheiden Mitglieder des Vorstands aus oder wurden zusätzlich weitere wählbare natürliche Personen, die auch Vereinsmitglieder sind, in den Vorstand aufgenommen, so haben diese kooptierten Vorstandsmitglieder für die restliche Funktionsperiode des Vorstandes von der der Kooptierung unmittelbar folgenden Generalversammlung mit absoluter Mehrheit der gültigen Stimmen bestätigt zu werden; bis zu einer solchen Generalversammlung dürfen maximal 3 (drei) Vorstände kooptiert werden, es sei denn, mangels Kooptierung würde die Mindestanzahl der Vorstandsmitglieder unterschritten werden.
Die Funktionsperiode des Vorstandes läuft weiter.
\item
Die Funktion des Präsidenten und Vizepräsidenten können durch Kooptierung nicht auf nicht gewählte Personen übertragen werden.
\item
Fällt der Vorstand ohne Selbstergänzung durch Kooptierung überhaupt oder auf unvorhersehbar lange Zeit aus, so ist jeder Rechnungsprüfer verpflichtet, unverzüglich eine außerordentliche Generalversammlung zum Zwecke der Neuwahl eines Vorstandes einzuberufen.
\end{enumerate}

(3)
Die Dauer der Funktionsperiode endet mit Ablauf jener - gemäß § 10 (1) jährlich innerhalb von 8 Monaten nach Beendigung eines Geschäftsjahres stattzufindenden - ordentlichen Generalversammlung, die über die Genehmigung des dritten Jahresabschlusses nach Wahl des Vorstandes entscheidet.

(4)
Der Vorstand wird für seine Sitzungen vom Präsidenten, bei dessen Verhinderung durch den Vizepräsidenten, schriftlich oder mündlich einberufen.

(5)
Der Vorstand ist beschlussfähig, wenn alle seine Mitglieder eingeladen wurden und mindestens die Hälfte der gewählten und kooptierten Vorstandsmitglieder und auch entweder der Präsident oder Vizepräsident und weiters mindestens gleich viel oder mehr gewählte bzw. infolge Kooptierung nachträglich von der Generalversammlung bestätigte Vorstandsmitglieder als noch nicht bestätigte kooptierte Vorstandsmitglieder anwesend sind.

(6)
Die Beschlüsse des Vorstandes werden in der Regel mit einfacher Stimmenmehrheit gefasst.
Bei Stimmengleichheit entscheidet die Stimme des Vorsitzenden.
Die Aufhebung oder Abänderung eines Vorstandsbeschlusses sowie der Beschluss über die Aufnahme oder den Ausschluss eines Mitgliedes (§ 7 (4)) bedürfen einer Mehrheit von zwei Drittel der Stimmen.

(7)
Außer durch Tod und Ablauf der Funktionsperiode erlischt die Funktion eines Vorstandsmitgliedes durch Enthebung durch die Generalversammlung und durch Rücktritt.
Vorstandsmitglieder, welche dreimal hintereinander den Vorstandssitzungen fernbleiben, können vom Vorstand ihres Mandates enthoben werden.

(8)
Wird die Zahl von drei Vorstandsmitgliedern unterschritten, ist der Rücktritt eines Vorstandsmitgliedes aufschiebend bedingt bis zu einer Kooptierung oder bis zum Ablauf jener Generalversammlung wirksam, die über eine Neuwahl entscheidet.

\subsection{§ 13
Aufgabenkreis des Vorstandes}

(1)
Dem Vorstand obliegt die Leitung des Vereines.
Er entscheidet und verfügt in allen Angelegenheiten, in denen die Statuten nichts anderes vorsehen und die nicht einem anderen Organ vorbehalten sind.

(2)
Insbesondere erstreckt sich die Tätigkeit auf

\begin{enumerate}
\item
die Anstellung eines Geschäftsführers;
\item
die Ernennung und Abberufung eines Sonderbeauftragten für Innovation, Forschung und Entwicklung;
\item
Überwachung der Geschäftsführung;
\item
die Verwaltung des Vereinsvermögens und die Sorge für eine geordnete Geldgebarung;
\item
die Einberufung und Vorbereitung von ordentlichen und außerordentlichen Generalversammlungen;
\item
Erlassung der Durchführungsbestimmungen für die Generalversammlung, insbesondere einer Wahlordnung;
\item
die Aufnahme, den Ausschluss und Streichung von Mitgliedern;
\item
die Anstellung und Kündigung von Angestellten des Vereines;
\item
Erstellung des Jahresvoranschlages;
\item
Abfassung des Rechenschaftsberichtes, umfassend auch die Geschäfte jener juristischen Person auf die eine Profiabteilung ausgegliedert wurde, und des Rechnungsabschlusses;
\item
Beschlussfassung über die Abgabe von Willenserklärungen des Vereins als Gesellschafter von jener juristischen Person, auf die eine Profiabteilung ausgegliedert wird bzw. wurde; dies vorbehaltlich, eine solche Entscheidung ist nicht Gegenstand eines Generalversammlungsbeschlusses;
\item
Beschlussfassung über Finanzierungsmaßnahmen des Vereins zugunsten jener juristischen Person, auf die eine Profiabteilung ausgegliedert wird bzw. wurde;
\end{enumerate}

(3)
Zur Regelung der inneren Organisation (Einsetzung von Ausschüssen, Geschäftsführung, Clubmanagement, u.a.m.) kann vom Leitungsorgan unter Berücksichtigung dieses Statutes eine Geschäftsordnung beschlossen werden.

\subsection{§ 14
Besondere Obliegenheiten einzelner und aller Vorstandsmitglieder}

(1)
Der Präsident repräsentiert den Verein nach außen.
Schriftstücke, insbesondere den Verein verpflichtende, sind vom Präsidenten und einem weiteren Vorstandsmitglied zu unterfertigen.
Im Innenverhältnis gilt folgendes:
Der Präsident führt die laufende Geschäftskorrespondenz, er führt den Vorsitz in den Generalversammlungen und in den Vorstandssitzungen.
Er hat die Einhaltung der Statuten zu überwachen und für die ordnungsgemäße Durchführung der gefassten Beschlüsse zu sorgen.

(2)
Im Verhinderungsfall wird der Präsident vom Vizepräsident vertreten.
Im Falle des Ausscheidens des Präsidenten wird der Verein von zwei Vorstandsmitgliedern bis zur nächsten Generalversammlung vertreten.

(3)
Der Schriftführer hat für die ordnungsgemäße Führung der Protokolle über die Generalversammlungen und die Vorstandssitzungen Sorge zu tragen.
Ihm obliegt die Verteilung der einlaufenden Schriftstücke und die Bearbeitung der Post, sofern hiefür nicht das Sekretariat zuständig ist.

(4)
Der Finanzreferent ist für die ordnungsgemäße Gebarung des Vereins zuständig.

(5)
Der Rechtsreferent vertritt den Verein in allen Rechtsfragen.

(6)
Rechtsgeschäfte zwischen Vorstandsmitgliedern und dem Verein bedürfen der Zustimmung eines anderen Vorstandsmitglieds.

(7)
Alle Vorstandsmitglieder sowie die Geschäftsführer der juristischen Peron, auf die die Profiabteilung ausgegliedert wird bzw. wurde, sind vorbehaltlich entschuldigter Verhinderung bei Generalversammlungen persönlich anwesend und stellen insbesondere bei Wahlen sich und ihr Programm den Mitgliedern persönlich vor.

\subsection{§ 15
Nicht belegt}

\subsection{§ 16
Nicht belegt}

\subsection{§ 17
Rechnungsprüfer}

(1)
Die beiden Rechnungsprüfer, die betriebswirtschaftliche Kenntnisse besitzen müssen, werden von der Generalversammlung für die Funktionsdauer des Vorstandes gewählt.
Eine Wiederwahl ist möglich.
Sie dürfen keinem Organ mit Ausnahme der Generalversammlung angehören, dessen Tätigkeit Gegenstand der Prüfung ist.

(2)
Es obliegt ihnen die laufende Geschäftskontrolle und die Überprüfung der gesamten Geldgebarungen sowie des Jahresabschlusses und der Ordnungsmäßigkeit der Rechnungslegung und der statutengemäßen Verwendung der Mittel.
Sie haben über das Ergebnis der Überprüfung der Generalversammlung zu berichten.

(3)
Nur die Rechnungsprüfer können in der Generalversammlung den Antrag auf Entlastung des Vorstandes stellen.

(4)
Erfüllt der Verein die Voraussetzung des § 22 Abs 2 VerG 2002, so gelten die Bestimmungen über die Rechnungsprüfer sinngemäß für den Abschlussprüfer.

\subsection{§ 18
Das Schiedsgericht}

(1)
In allen aus dem Vereinsverhältnis entstehenden Streitigkeiten entscheidet das Schiedsgericht, dies ist eine Schlichtungseinrichtung im Sinne des VerG 2002 und kein Schiedsgericht nach den §§ 577 ff ZPO.
Das Schiedsgericht setzt sich aus 5 ordentlichen Mitgliedern zusammen.
Es wird derart gebildet, dass jeder Streitteil über Aufforderung durch den Vorstand innerhalb von 14 Tagen dem Vorstand 2 ordentliche Mitglieder als Schiedsrichter namhaft macht.
Diese wählen binnen weiteren 14 Tagen mit Stimmenmehrheit einen Vorsitzenden.
Bei Stimmengleichheit entscheidet das Los.

(2)
Die Einberufung des Schiedsgerichts hat unverzüglich durch den Vorsitzenden zu erfolgen.
Das Schiedsgericht entscheidet, ohne an bestimmte Vorschriften gebunden zu sein, nach bestem Wissen und Gewissen.
Es trifft seine Entscheidungen bei Anwesenheit aller seiner Mitglieder mit einfacher Stimmenmehrheit.
Das Schiedsgericht hat auf eine zügige Bearbeitung und einen möglichst schnellen Abschluss des Schiedsverfahrens zu achten.
Die Entscheidungen sind vereinsintern endgültig.

(3)
Mitglieder, die sich in einer Streitigkeit aus dem Vereinsverhältnis nicht dem Schiedsgericht unterwerfen oder dessen Entscheidungen nicht anerkennen, können aus dem Verein ausgeschlossen werden.

\section{DRITTES HAUPTSTÜCK\\Das demokratische Leben des FC Wacker Innsbruck}

\subsection{§ 19
Demokratieverständnis}

(1)
Organisation und Arbeitsweise des Mitgliedervereins "`FC Wacker Innsbruck"' beruhen auf dem Grundsatz der repräsentativen Demokratie.
Die gemeinschaftliche Willensbildung gründet auf der Selbstbestimmung seiner Mitglieder und verfolgt das Ziel der Selbstbestimmtheit des Vereins.

(2)
Der Verein achtet in seinem gesamten Handeln den Grundsatz der Gleichheit seiner Mitglieder.
Diese genießen ein gleiches Maß an Aufmerksamkeit seitens der Organe des Vereins.

(3)
Die Organe des Vereins geben den Mitgliedern in geeigneter Weise die Möglichkeit, ihre Ansichten mit ihnen auszutauschen und öffentlich bekannt zu geben.
Die Organe des Vereins pflegen insbesondere auf Vereinsabenden einen offenen, transparenten und regelmäßigen Dialog mit den Mitgliedern.
Die Entscheidungen werden so offen und so mitgliedernah wie möglich getroffen.

\subsection{§ 20
Wahl des Vorstandes}

(1)
Der Präsident und die sonstigen Mitglieder des Vorstandes werden auf der Generalversammlung in getrennten Wahlgängen bestimmt.
Zunächst der Präsident im Verfahren der Einzelwahl, sodann ein von ihm unter Angabe der Funktionen vorgeschlagenes Team sonstiger Vorstandsmitglieder im Verfahren der Blockwahl.
Die Mitglieder des Vorstands gelten erst dann als bestellt, wenn sowohl Präsident als auch Team gewählt sind.
Die Wahlen sind in jedem Fall geheim durchzuführen.

(2)
Das Wahlverfahren vor der Generalversammlung gliedert sich wie folgt:

\begin{enumerate}[label=\alph*)]
\item
Für die Funktion des Präsidenten ist beim Vorstand, organisatorisch vertreten durch das Clubmanagement, ein schriftlicher Wahlvorschlag einzubringen.
Jedes Mitglied ist berechtigt, ein ordentliches Mitglied zu nominieren.
Der Wahlvorschlag hat den vollständigen Namen, Meldeanschrift, Geburtsdatum und Geburtsort sowie die eigenhändige Unterschrift sowohl des Vorgeschlagenen als auch des (der) Vorschlagenden zu enthalten.
Die Nominierungsfrist endet zwei Wochen vor der Generalversammlung.
\item
Der nominierte Präsidentschaftskandidat hat den Wahlvorschlag um sein Team der sonstigen Vorstandsmitglieder (Teamvorschlag) zu ergänzen;
hierbei ist insbesondere der zu wählende Vizepräsident namentlich festzulegen.
Voraussetzung ist deren ordentliche Mitgliedschaft.
Der Ergänzungsvorschlag hat den vollständigen Namen, Meldeanschrift, Geburtsdatum und Geburtsort sowie die eigenhändige Unterschrift sowohl der Vorgeschlagenen als auch des Vorschlagenden zu enthalten.
Die Ergänzungsfrist endet eine Woche vor der Generalversammlung.
\item
Das Clubmanagement hat die Mitglieder mit Ablauf der jeweiligen Fristen über die eingebrachten Vorschläge zu informieren.
Liegen mit Ablauf der jeweiligen Fristen keine Vorschläge vor, so kann der Vorstand eine Nachfrist setzen.
\item
Eine Woche vor der Generalversammlung ist ein Vereinsabend durchzuführen.
Den durch Wahlvorschlag nominierten Kandidaten ist dabei Gelegenheit zur Präsentation einzuräumen.
Ab dem Vereinsabend kann der nominierte Präsidentschaftskandidat seinen Teamvorschlag bis zur Durchführung der Blockwahl auf der Generalversammlung jederzeit ändern.
\end{enumerate}

(3)
Das Wahlverfahren auf der Generalversammlung steht unter der Leitung des Vorstands und gliedert sich wie folgt:

\begin{enumerate}[label=\alph*)]
\item
Über den Präsidenten wird gemäß Absatz 1 persönlich abgestimmt.
Es gilt jener Kandidat als gewählt, der die Mehrheit der gültigen Stimmen erhält.
Stellt sich nur ein Kandidat der Wahl, so muss er die absolute Mehrheit der gültigen Stimmen auf sich vereinigen können.
\item
Über die sonstigen Mitglieder des Vorstandes (Teamvorschlag) wird gemäß Absatz 1 im Block abgestimmt.
Die Wahl ist als Abstimmung mit den Wahlmöglichkeiten "`ja"' und "`nein"' durchzuführen.
Der Teamvorschlag gilt als angenommen, wenn er die absolute Mehrheit der gültigen Stimmen erhält.
\item
Erhält der Teamvorschlag nicht die erforderliche Mehrheit, so kann der bereits gewählte Präsidentschaftskandidat diesen entweder
	\begin{itemize}
	\renewcommand\labelitemi{-}
	\item
	ändern und erneut zur Abstimmung gemäß Buchstabe b) stellen oder
	\item
	von seiner Wahl zurücktreten.
	\end{itemize}
\item
Das Recht zur Änderung des Teamvorschlags gemäß § 20 Abs 3 lit c Z 1 kann nur einmal in Anspruch genommen werden.
Führt das Wahlverfahren auf der Generalversammlung zu keinem Ergebnis, so ist jeder Rechnungsprüfer verpflichtet, unverzüglich eine außerordentliche Generalversammlung zum Zwecke der Wahl eines Vorstandes einzuberufen.
Diese hat jedenfalls innerhalb einer Frist von 42 Tagen stattzufinden.
Das Wahlverfahren startet erneut.
\item
Der amtierende Vorstand bleibt so lange im Amt, bis ein neuer Vorstand gewählt ist.
\end{enumerate}

\subsection{§ 21
Wahl der Rechnungsprüfer}

Über die Rechnungsprüfer wird in persönlicher Wahl einzeln abgestimmt.
Es gilt jener Kandidat als gewählt, der die Mehrheit der gültigen Stimmen erhält.
Stellt sich nur ein Kandidat der Wahl, so muss er die absolute Mehrheit der gültigen Stimmen auf sich vereinigen können.

\section{VIERTES HAUPTSTÜCK\\Schlussbestimmungen}

\subsection{§ 22
Begriffliches Verständnis}

Um Verständlichkeit und Lesbarkeit der vorliegenden Statuten zu gewährleisten, sind die Darstellung und Beschreibung der Personen und Funktionen sowohl in der männlichen als auch in der weiblichen Form zu verstehen.

\subsection{§ 23
Auflösung}

(1)
Die freiwillige Auflösung des Vereines kann nur in einer zu diesem Zweck einberufenen außerordentlichen Generalversammlung und nur mit einer Mehrheit von zwei Drittel der gültigen Stimmen beschlossen werden.

(2)
Diese Generalversammlung hat auch – sofern Vereinsvermögen vorhanden ist – über die Abwicklung zu beschließen.
Insbesondere hat sie einen Abwickler zu berufen und Beschluss darüber zu fassen, wem dieser das nach Abdeckung der Passiven verbleibende Vermögen zu übertragen hat.

(3)
Bei Auflösung des Vereins oder bei Wegfall des bisherigen begünstigten Vereinszwecks darf das allenfalls vorhandene Vereinsvermögen in keiner wie immer gearteten Form den Vereinsmitgliedern zugutekommen, sondern ist für sportliche gemeinnützige Zwecke im Sinne der §§ 34ff BAO zu verwenden.

\end{document}
