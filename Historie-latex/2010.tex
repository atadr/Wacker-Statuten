\documentclass[10pt,a4paper,landscape]{article}
\usepackage[utf8]{inputenc}
\usepackage{amsmath}
\usepackage{amsfonts}
\usepackage{amssymb}
\usepackage{lmodern}
\title{Statuten des FC Wacker Innsbruck}
\begin{document}

\section*{Entwurf zu den Statuten des FC Wacker Innsbruck}
Fassung 2010

\section*{Inhalt:}


\section*{Präambel}

Geleitet von dem Willen der Mitglieder, ihre Zukunft gemeinsam zu gestalten, begründen diese Statuten sowie das Markengrundbuch den rechtlichen und inhaltlichen Rahmen des FC Wacker Innsbruck.
Unser Verein steht allen offen, die seine Werte und Grundsätze achten und im positiven Sinne nach Außen tragen.

In Übereinstimmung mit allen verantwortlichen Personen, denen das Wohlergehen und die nachhaltige Weiterentwicklung des offenen, demokratischen Mitgliedervereines FC Wacker Innsbruck am Herzen liegen, wurde das Markengrundbuch erarbeitet und niedergeschrieben.
Das Markengrundbuch erfährt nunmehr jenen vorrangigen Stellenwert, der es allen Beteiligten ermöglicht, nach den beschriebenen Inhalten und Werten zu leben, zu handeln und diese auch von anderen einzufordern.
Damit bringt der FC Wacker Innsbruck seine Einzigartigkeit zum Ausdruck.

Auf Grundlage dieser Werte begreift der Verein den Fußballsport und sich selbst als Kulturgut und entwickelt daraus seine spezifische Vereinskultur (Markengrundbuch - Version 2009 in seiner jeweils geltenden Fassung).

Im Verständnis und im Bewusstsein, dass diese Statuten die rechtliche Verkörperung unserer Wertekultur bilden, kommen wir, die Mitglieder, wie folgt überein:

\section*{Erstes Hauptstück: Die äußere und innere Identität des FC Wacker Innsbruck}

\subsection{§ 1 Name, Sitz und Tätigkeitsbereich}

(1) Der Verein führt den Namen "FC Wacker Innsbruck".


\end{document}