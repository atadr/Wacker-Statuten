\documentclass[10pt,a4paper]{article}
\usepackage[a4paper, total={6.5in, 9.5in}]{geometry}
\usepackage[ngerman]{babel}
\usepackage{hyphenat}
\usepackage{lmodern}
\usepackage{parskip}
\usepackage{enumitem}

\author{FC Wacker Innsbruck}
\title{Statuten des FC Wacker Innsbruck}
\begin{document}
\setlength{\parskip}{25pt}
\setcounter{secnumdepth}{0}

\centering
\itshape

\section{Entwurf \\ zu den \\ Statuten des FC Wacker Innsbruck}

Fassung 2010

\vspace{5ex}

\section{Inhalt:}

\vspace{10ex}

Präambel

Erstes Hauptstück: Die äußere und innere Identität des FC Wacker Innsbruck

Zweites Hauptstück: Die institutionelle Architektur des FC Wacker Innsbruck - die Organe und ihre Wirkungsweise

Drittes Hauptstück: Das demokratische Leben des FC Wacker Innsbruck

Viertes Hauptstück: Schlussbestimmungen

\clearpage

\section{Präambel}

\flushleft
\vspace{10ex}


Geleitet von dem Willen der Mitglieder, ihre Zukunft gemeinsam zu gestalten, begründen diese Statuten sowie das Markengrundbuch den rechtlichen und inhaltlichen Rahmen des FC Wacker Innsbruck.
Unser Verein steht allen offen, die seine Werte und Grundsätze achten und im positiven Sinne nach Außen tragen.

In Übereinstimmung mit allen verantwortlichen Personen, denen das Wohlergehen und die nachhaltige Weiterentwicklung des offenen, demokratischen Mitgliedervereines FC Wacker Innsbruck am Herzen liegen, wurde das Markengrundbuch erarbeitet und niedergeschrieben.
Das Markengrundbuch erfährt nunmehr jenen vorrangigen Stellenwert, der es allen Beteiligten ermöglicht, nach den beschriebenen Inhalten und Werten zu leben, zu handeln und diese auch von anderen einzufordern.
Damit bringt der FC Wacker Innsbruck seine Einzigartigkeit zum Ausdruck.

Auf Grundlage dieser Werte begreift der Verein den Fußballsport und sich selbst als Kulturgut und entwickelt daraus seine spezifische Vereinskultur (Markengrundbuch - Version 2009 in seiner jeweils geltenden Fassung).

Im Verständnis und im Bewusstsein, dass diese Statuten die rechtliche Verkörperung unserer Wertekultur bilden, kommen wir, die Mitglieder, wie folgt überein:

\clearpage
\setlength{\parskip}{10pt}
\upshape

\section{Erstes Hauptstück: Die äußere und innere Identität des FC Wacker Innsbruck}

\subsection{§ 1
Name, Sitz und Tätigkeitsbereich}

(1)
Der Verein führt den Namen "`FC Wacker Innsbruck"'.

(2)
Er hat seinen Sitz in Innsbruck und erstreckt seine Tätigkeit auf das gesamte Gebiet der Europäischen Union (EU).
Die Sportausübung ist auch außerhalb der EU möglich.

(3)
Die Errichtung von Zweigvereinen ist nicht beabsichtigt.

\subsection{§ 2
Vereinsfarben und Vereinswappen}

(1)
Die Vereinsfarben sind "`Schwarz-Grün"'.

(2)
Das Vereinswappen visualisiert den Ursprung, die Identität und die Zukunft des FC Wacker Innsbruck.
Das Ursprungswappen ist eingebettet in das schwarz-grüne Selbstverständnis und mit dem goldenen Band des Sieges umrahmt.

\subsection{§ 3
Zweck}

(1)
Der Verein, dessen Tätigkeit nicht auf Gewinn gerichtet ist, bezweckt die Pflege und Förderung des Körpersportes jeder Art, insbesondere die Ausübung des Nachwuchs-, Amateur- und Spitzenfußballs sowie aller damit unmittelbar und mittelbar verbundenen Aufgaben, unter Ausschluss jeder politischen Tendenz.

(2)
Der Verein wird auf gemeinnütziger Basis von einem gewählten Vorstand ehrenamtlich geführt.

(3)
Mitarbeiter oder Mitglieder von Organen von Unternehmen, die zu mehreren Vereinen oder Tochtergesellschaften der Lizenzligen bzw. Muttervereinen in vertraglicher Beziehungen stehen, dürfen nicht Mitglied in Kontroll-, Geschäftsführungs- und Vertretungsorganen sein, es sei denn, ein wirtschaftliches Interesse dieser Unternehmen an sportlichen Erfolgen mehrerer Vereine oder Tochtergesellschaften der Lizenzligen oder Muttervereinen ist auszuschließen.

(4)
Mitglieder von Kontroll-, Geschäftsführungs- und Vertretungsorganen anderer Vereine oder anderen Tochtergesellschaften der Lizenzligen oder anderer Muttervereine dürfen keine Funktionen in Organen des Vereins übernehmen.

(5)
Der Verein erfüllt den Vereinszweck durch die Teilnahme an verschiedenen Fußball-Meisterschaften, sowie durch Pflege des Fußballsportes auf allen Gebieten des Breiten-, Spitzen- und esundheitssportes.
Auch die sportliche Ausbildung und Weiterbildung der Mitglieder ist Ziel des Vereines.

\subsection{§ 4
Mittel zur Erreichung des Vereinszweckes}

(1)
Der Vereinszweck soll durch die in den Absätzen 2 und 3 angeführten ideellen und materiellen Mittel erreicht werden.

(2)
Als ideelle Mittel dienen

\begin{enumerate}[label=\alph*)]
\item
Maßnahmen zur Umsetzung des Markengrundbuchs (Version 2009 idgF.)
\item
sportliche und gesellige Veranstaltungen
\item
Vereinsabende und sonstige Maßnahmen der Mitglieder- und Fanbetreuung
\item
Vorträge und Versammlungen, Symposien und Diskussionsforen
\item
Herausgabe einer Stadionzeitung und sonstiger Publikationen
\item
Homepage
\item
Aufbau eines wissenschaftlichen Think-Tank
\item
Aufbau einer Fachbibliothek, eines Museums sowie eines Dokumentationszentrums
\item
Aufbau eines Trainingszentrums und einer Akademie
\end{enumerate}

(3)
Die erforderlichen materiellen Mittel des Vereins sollen aufgebracht werden durch

\begin{enumerate}[label=\alph*)]
\item
Mitgliedsbeiträge
\item
Erträgnisse aus Veranstaltungen des Vereins
\item
Sponsorbeiträge
\item
Erträgnisse sonstiger finanzieller Verwertung (z.B. Sportanlagen)
\item
Erträgnisse aus Beteiligung an Kapitalgesellschaften
\item
Erträgnisse aus Publikationen des Vereins und Förderungen
\item
Spenden, Vermächtnisse und sonstige Zuwendungen
\end{enumerate}

(4)
Die Vereinsgelder dürfen nur zur Deckung von Ausgaben für Vereinszwecke verwendet werden.
Die Mittel des Vereines dürfen ausschließlich für die begünstigten Zwecke verwendet werden.
Die Mitglieder des Vereines dürfen keine Gewinnanteile und in ihrer Eigenschaft als Mitglieder auch keine sonstigen Zuwendungen aus Mitteln des Vereines erhalten.
Es darf keine Person durch Verwaltungsausgaben, die dem Zweck des Vereines fremd sind oder durch unverhältnismäßig hohe Vergütungen begünstigt werden.

(5)
Der Verein ist berechtigt, die zur sportlichen Verwertung seiner Anlagen allenfalls erforderlichen Gewerbe nach vorheriger Einholung der behördlichen Genehmigung zur Ausübung der Gewerbeberechtigung anzumelden und auszuüben.
Die Einnahmen dürfen nur im Sinne des Vereinszweckes verwendet werden.

\subsection{§ 5
Arten der Mitgliedschaft}

(1)
Der demokratische Mitgliederverein "`FC Wacker Innsbruck"' gliedert sich in

\begin{enumerate}[label=\alph*)]
\item
ordentliche Mitglieder
\item
fördernde Mitglieder
\item
Ehrenmitglieder
\end{enumerate}

(2)
Arten der Mitgliedschaft:

\begin{enumerate}[label=\alph*)]
\item
Ordentliche Mitglieder sind jene, die aufgrund einer Beitrittserklärung einen jährlichen Mitgliedsbeitrag leisten.
Voraussetzung ist die Volljährigkeit.
Dieser wird von der Generalversammlung festgelegt und richtet sich nach Alter und Lebensabschnitt.
\item
Fördernde Mitglieder sind jene, die die Zwecke des Vereins durch Zahlung eines jährlichen Beitrags in der Höhe, die ebenfalls von der Generalversammlung festgelegt wird, oder durch sonstige Zuwendungen jeglicher Art fördern.
\item
Ehrenmitglieder sind Personen, die hierzu wegen besonderer Verdienste um den Verein ernannt werden.
\end{enumerate}

\subsection{§ 6
Erwerb der Mitgliedschaft}

(1)
Mitglieder des Vereins können nur physische Personen werden.

(2)
Über die Aufnahme der Mitglieder entscheidet der Vorstand mit einfacher Mehrheit.
Die Ablehnung einer Aufnahme kann ohne Angabe von Gründen erfolgen.

(3)
Die Ernennung zum Ehrenmitglied erfolgt auf Antrag des Vorstandes durch die Generalversammlung.

(4)
Ist der Mitgliedswerber noch nicht volljährig, so bedarf die Beitrittserklärung der Zustimmung seines gesetzlichen Vertreters.

\subsection{§ 7
Beendigung der Mitgliedschaft}

(1)
Die Mitgliedschaft erlischt durch Tod, durch freiwilligen Austritt, durch Streichung und durch Ausschluss.

(2)
Der freiwillige Austritt kann nur zum 30. Juni jedes Jahres erfolgen, die Austrittserklärung muss bis längstens 1. Juni desselben Jahres dem Vorstand des Vereines zugegangen sein.
Der Austritt entbindet nicht von der Erfüllung der bis zum Austrittszeitpunkt entstandenen Verbindlichkeiten gegenüber dem Verein.

(3)
Die Streichung eines Mitgliedes kann der Vorstand vornehmen, wenn dieses bis spätestens 1. 12. des laufenden Geschäftsjahres mit der Zahlung des Mitgliedsbeitrages trotz Mahnung im Rückstand ist.
Gegen diese Streichung steht den Mitgliedern innerhalb eines Monats das Recht der Berufung an die nächste Generalversammlung zu.
Die Streichung entbindet ebenfalls nicht von der Erfüllung der bis zum Streichungszeitpunkt entstandenen Verbindlichkeiten gegenüber dem Verein.

(4)
Der Ausschluss eines Mitgliedes aus dem Verein kann vom Vorstand wegen grober Verletzung der Mitgliedspflichten und wegen unehrenhaften und vereinsschädigenden Verhaltens verfügt werden.
Gegen den Ausschluss ist die Berufung an die Generalversammlung binnen einem Monat zulässig, bis zu deren Entscheidung die Mitgliedsrechte ruhen.

(5)
Die Aberkennung der Ehrenmitgliedschaft kann aus den in Abs 4 genannten Gründen von der Generalversammlung auf Antrag des Vorstandes beschlossen werden.

\subsection{§ 8
Rechte und Pflichten der Mitglieder}

(1)
Allen Mitgliedern steht die freie und tätige Mitwirkung am Vereinsleben im Rahmen der Vereinsstatuten offen.
Sie üben ihre Teilhabe auf der Grundlage von Offenheit, Respekt und Verantwortung aus.

\begin{enumerate}[label=\alph*)]
\item
Nur ordentliche Mitglieder, deren Mitgliedschaft seit mindestens einem Jahr ununterbrochen besteht, haben das aktive Wahlrecht sowie das Stimmrecht in der Generalversammlung.
Das Erfordernis der ununterbrochenen einjährigen Mitgliedschaft ist auch dann erfüllt, wenn diese entweder zur Gänze oder in Teilen fördernder Natur gewesen ist.
\item
Nur ordentliche Mitglieder haben das passive Wahlrecht.
\item
Für die Erlangung des aktiven und des passiven Wahlrechts sowie des Stimmrechts in der Generalversammlung ist die Volljährigkeit Voraussetzung.
Fördernde Mitglieder können ihre Mitgliedschaft jederzeit in eine ordentliche umwandeln.
\end{enumerate}

(2)
Alle Mitglieder haben regelmäßige Mitgliedsbeiträge zu leisten.
Diese werden von der Generalversammlung festgesetzt und sind im Vorhinein zu entrichten, jedoch bis spätestens 30.6. eines jeden Jahres.

(3)
Alle Mitglieder sind verpflichtet, die Interessen des Vereins nach besten Kräften zu fördern, seine Werte zu achten und alles zu unterlassen, wodurch Ansehen und Zweck des Vereins leiden könnten.
Sie haben die Statuten und ihre Durchführungsbestimmungen sowie die Beschlüsse der Vereinsorgane genau zu beachten.

(4)
Die Mitglieder verpflichten sich jeglichem rassistischen Verhalten im Stadion und im Klub entschieden entgegenzutreten, sowie zur Förderung von ethnischen Minderheiten und Migranten im Verein und zur Zusammenarbeit mit allen Organisationen, welche dem Problem des Rassismus im Fußball entgegenwirken.
Insbesondere Aktivitäten im Rahmen des Netzwerkes "`FARE - Football Against Racism in Europe"' und der österreichischen Kampagne "`FairPlay - Viele Farben. Ein Spiel"' sollen unterstützt werden.
Außerdem anerkennen die Mitglieder die Grundsätze des Zehn Punkte Plans der UEFA zur Bekämpfung des Rassismus im Fußball.

\clearpage

\section{Zweites Hauptstück: Die institutionelle Architektur des FC Wacker Innsbruck – die Organe und ihre Wirkungsweise}

\subsection{§ 9
Organe des Vereines}
(1)
Die Organe des Vereines sind:

\begin{enumerate}[label=\alph*)]
\item
die Generalversammlung
\item
der Vorstand
\item
der Wirtschaftsrat
\item
die Rechnungsprüfer
\item
das Schiedsgericht
\end{enumerate}

(2)
Jedes Organ handelt nach Maßgabe der ihm in diesen Statuten zugewiesenen Zuständigkeiten und Wirkungsweisen.
Die Organe arbeiten loyal und koordiniert zusammen.

\subsection{§ 10
Die Generalversammlung}

(1)
Die ordentliche Generalversammlung findet einmal jährlich innerhalb von 8 Monaten nach Beendigung des Geschäftsjahres statt.
Sie wird durch den Vorstand einberufen.

(2)
Eine außerordentliche Generalversammlung ist innerhalb von 21 Tagen vom Vorstand einzuberufen:

\begin{enumerate}[label=\alph*)]
\item
auf Beschluss des Vorstandes
\item
auf Antrag der Rechnungsprüfer
\item
auf Antrag eines Zehntels der Mitglieder.
\end{enumerate}

(3)
In den Fällen von §§ 12 Abs 2 lit c und 20 Abs 3 lit d obliegt die Einberufung den Rechnungsprüfern.
Der Antrag auf eine außerordentliche Generalversammlung hat gleichzeitig jene(n) Antrag/Anträge zu enthalten, die bei der außerordentlichen Generalversammlung auf die Tagesordnung gebracht werden sollen.
Die Antragspunkte müssen der Beschlussfassung der Generalversammlung unterliegen, andernfalls gilt der Antrag als nicht gestellt.
Ein Antrag ohne Angabe von Gründen ist nichtig.

(4)
Verstreicht trotz ordnungsgemäßem Antrag auf Einberufung einer außerordentlichen Generalversammlung die Frist, ohne dass der Vorstand entsprechend tätig wird, so sind diejenigen, die den Antrag ordnungsgemäß eingebracht haben, berechtigt, unter Wahrung der vorgenannten Formen und Fristen die Generalversammlung selbst auf Kosten des Vereins einzuberufen.

(5)
Die außerordentliche Generalversammlung hat innerhalb von 42 Tagen ab Antrag oder Beschluss stattzufinden.
Der Tag der ordentlichen bzw. außerordentlichen Generalversammlung ist allen Mitgliedern mindestens 21 Tage vorher schriftlich oder durch eine Einschaltung in den örtlichen Medien bekannt zu geben.
Die Anberaumung der Generalversammlung hat unter Angabe der vorläufigen Tagesordnung zu erfolgen.

(6)
Die Mitglieder haben die Möglichkeit, bis 14 Tage vor der Generalversammlung, Anträge an die Generalversammlung zu stellen.

(7)
Anträge, die erst in der Versammlung gestellt werden, werden nur behandelt, wenn der Vorstand der Behandlung zustimmt oder die Versammlung die Behandlung mit Dreiviertelmehrheit beschließt.

(8)
Bei der Generalversammlung sind nur die ordentlichen Mitglieder stimmberechtigt, wobei jedes Mitglied nur eine Stimme hat.
Die Teilnahme an der Generalversammlung steht allen Vereinsmitgliedern zu.
Die Übertragung des Stimmrechts auf ein anderes Mitglied im Wege einer schriftlichen Bevollmächtigung ist nicht zulässig.

(9)
Die Wahlen und Beschlussfassungen erfolgen in der Regel mit einfacher Stimmenmehrheit.
Bei Stimmengleichheit gilt der Antrag als abgelehnt.

(10)
Den Vorsitz in der Generalversammlung führt der Obmann, in dessen Verhinderung ein anderes Mitglied des Vorstandes.

\subsection{§ 11
Aufgabenkreis der Generalversammlung}

Der Generalversammlung, die das oberste Beschlussorgan des Vereins ist, sind folgende Aufgaben vorbehalten:
\begin{enumerate}
\item
Entgegennahme und Genehmigung des Rechenschaftsberichtes des Vorstandes und des Jahresabschlusses unter Einbindung der Rechnungsprüfer
\item
Beschlussfassung über Voranschlag
\item
Wahl und Enthebung der Mitglieder des Vorstandes und der Rechnungsprüfer, wobei letztere keinem weiteren Organ als der Generalversammlung angehören dürfen
\item
Entlastung des Vorstandes
\item
Beschlussfassung über die Höhe der Mitgliedsbeiträge.
\item
Entscheidungen über Berufungen gegen Ausschlüsse von der Mitgliedschaft
\item
Beschlussfassung über Änderungen der Statuten und die freiwillige Auflösung des Vereins.
Eine Statutenänderung ist in einer Generalversammlung mit einer Mehrheit von zwei Drittel der gültigen Stimmen zu beschließen
\item
Beschlussfassung über alle Maßnahmen, die geeignet sind, die Befugnisse der Generalversammlung zu beeinträchtigen, die Profiabteilung in eine eigene Rechtsform überzuführen oder eine Lizenzübertragung zur Teilnahme an der österreichischen Bundesliga zu bewirken.
Die Beschlussfassung darüber erfordert die Mehrheit von zwei Drittel der gültigen Stimmen in einer Generalversammlung.
\item
Beratung und Beschlussfassung über sonstige auf der Tagesordnung stehende Fragen
\item
Entgegennahme des Tätigkeitsberichtes des Wirtschaftsrates
\item
Genehmigung von Rechtsgeschäften zwischen Rechnungsprüfern und dem Verein
\item
Verleihung und Aberkennung der Ehrenmitgliedschaft
\end{enumerate}

\subsection{§ 12
Der Vorstand}

(1)
Der Vorstand wird durch Wahl gemäß § 20 bestellt und besteht aus mindestens 5 Mitgliedern.
Diese sind:

\begin{enumerate}[label=\alph*)]
\item
der Obmann
\item
der Obmannstellvertreter
\item
der Schriftführer
\item
der Finanzreferent
\item
der Rechtsreferent
\end{enumerate}

(2)
Der Vorstand hat das Recht, bei Ausscheiden eines Vorstandsmitgliedes an seine Stelle ein anderes wählbares Vereinsmitglied in den Vorstand zu kooptieren, dies bis zur ächstfolgenden Generalversammlung.

\begin{enumerate}[label=\alph*)]
\item
Scheidet der Obmann aus, so hat auf dieser Generalversammlung ein neuer Vorstand mit neuer Funktionsperiode gemäß § 20 gewählt zu werden.
\item
Scheiden sonstige Mitglieder des Vorstands aus, so haben die kooptierten Vorstandsmitglieder von der Generalversammlung mit absoluter Mehrheit der gültigen Stimmen bestätigt zu werden.
Die Funktionsperiode des Vorstandes läuft weiter.
\item
Fällt der Vorstand ohne Selbstergänzung durch Kooptierung überhaupt oder auf unvorhersehbar lange Zeit aus, so ist jeder Rechnungsprüfer verpflichtet, unverzüglich eine außerordentliche Generalversammlung zum Zwecke der Neuwahl eines Vorstandes einzuberufen.
\end{enumerate}

(3)
Die Dauer der Funktionsperiode beträgt 3 Jahre.

(4)
Der Vorstand wird für seine Sitzungen vom Obmann bzw. dem Finanzreferenten schriftlich oder mündlich einberufen.

(5)
Der Vorstand ist beschlussfähig, wenn alle seine Mitglieder eingeladen wurden und mindestens die Hälfte anwesend ist.

(6)
Die Beschlüsse des Vorstandes werden mit einfacher Stimmenmehrheit gefasst.
Bei Stimmengleichheit entscheidet die Stimme des Vorsitzenden.
Die Aufhebung oder Abänderung eines Vorstandsbeschlusses sowie der Beschluss über die Aufnahme oder den Ausschluss eines Mitgliedes bedürfen einer Mehrheit von zwei Drittel der Stimmen.

(7)
Außer durch Tod und Ablauf der Funktionsperiode erlischt die Funktion eines Vorstandsmitgliedes durch Enthebung durch die Generalversammlung und durch Rücktritt.
Vorstandsmitglieder, welche dreimal hintereinander den Vorstandssitzungen fernbleiben, können vom Vorstand ihres Mandates enthoben werden.

\subsection{§ 13
Aufgabenkreis des Vorstandes}

(1)
Dem Vorstand obliegt die Leitung des Vereines.
Er entscheidet und verfügt in allen Angelegenheiten, in denen die Statuten nichts anderes vorsehen und die nicht einem anderen Organ vorbehalten sind.

(2)
Insbesondere erstreckt sich die Tätigkeit auf
\begin{itemize}
\renewcommand\labelitemi{-}
\item
die Anstellung eines Geschäftsführers
\item
die Ernennung und Abberufung eines Sonderbeauftragten für Innovation, Forschung und Entwicklung
\item
Überwachung der Geschäftsführung
\item
die Verwaltung des Vereinsvermögens und die Sorge für eine geordnete Geldgebarung
\item
die Einberufung und Vorbereitung von ordentlichen und außerordentlichen Generalversammlungen
\item
Erlassung der Durchführungsbestimmungen für die Generalversammlung, insbesondere einer Wahlordnung
\item
die Aufnahme, den Ausschluss und Streichung von Mitgliedern
\item
die Anstellung und Kündigung von Angestellten des Vereines
\item
Erstellung des Jahresvoranschlages
\item
Abfassung des Rechenschaftsberichtes und des Rechnungsabschlusses
\end{itemize}

(3)
Zur Regelung der inneren Organisation (Einsetzung von Ausschüssen, Geschäftsführung, Clubmanagement, u.a.m.) kann vom Leitungsorgan unter Berücksichtigung dieses Statutes eine Geschäftsordnung beschlossen werden.

\subsection{§ 14
Besondere Obliegenheiten einzelner und aller Vorstandsmitglieder}

(1)
Der Obmann repräsentiert den Verein nach außen.
Schriftstücke, insbesondere den Verein verpflichtende, sind vom Obmann und einem weiteren Vorstandsmitglied zu unterfertigen.
Im Innenverhältnis gilt folgendes:
Der Obmann führt die laufende Geschäftskorrespondenz, er führt den Vorsitz in den Generalversammlungen und in den Vorstandssitzungen.
Er hat die Einhaltung der Statuten zu überwachen und für die ordnungsgemäße Durchführung der gefassten Beschlüsse zu sorgen.

(2)
Im Verhinderungsfall wird der Obmann vom Obmannstellvertreter vertreten.
Im Falle des Ausscheidens des Obmanns wird der Verein von zwei Vorstandsmitgliedern bis zur nächsten Generalversammlung vertreten.

(3)
Der Schriftführer hat für die ordnungsgemäße Führung der Protokolle über die Generalversammlungen und die Vorstandssitzungen Sorge zu tragen.
Ihm obliegt die Verteilung der einlaufenden Schriftstücke und die Bearbeitung der Post, sofern hiefür nicht das Sekretariat zuständig ist.

(4)
Der Finanzreferent ist für die ordnungsgemäße Gebarung des Vereins zuständig.

(5)
Der Rechtsreferent vertritt den Verein in allen Rechtsfragen.

(6)
Rechtsgeschäfte zwischen Vorstandsmitgliedern und dem Verein bedürfen der Zustimmung eines anderen Vorstandsmitglieds.

(7)
Alle Vorstandsmitglieder sind vorbehaltlich entschuldigter Verhinderung bei Generalversammlungen persönlich anwesend und stellen insbesondere bei Wahlen sich und ihr Programm den Mitgliedern persönlich vor.

\subsection{§ 15
Der Wirtschaftsrat}

(1)
Der Wirtschaftsrat erfüllt eine beratende Aufgabe.
Er besteht aus mindestens drei Mitgliedern.
Jeder Sponsor, der einen jährlichen Sponsorbeitrag von mindestens € 150.000,-- an den Verein leistet, hat das Recht, einen Vertreter in den Wirtschaftsrat zu entsenden.
Jeder Sponsor, der einen jährlichen Sponsorbeitrag von mindestens € 300.000,-- an den Verein leistet, hat das Recht, zwei Vertreter in den Wirtschaftsrat zu entsenden.

(2)
Der Wirtschaftsrat hat aus seiner Mitte folgende Funktionäre zu wählen

\begin{enumerate}[label=\alph*)]
\item
den Vorsitzenden des Wirtschaftsrates
\item
den Stellvertreter des Vorsitzenden des Wirtschaftsrates
\end{enumerate}

(3)
Zur Bestellung der einzelnen Funktionäre ist die Anwesenheit von mindestens zwei Drittel der Mitglieder des Wirtschaftsrates erforderlich.
Es wird zuerst der Vorsitzende des Wirtschaftsrates gewählt und bei Stimmengleichheit entscheidet das Los.

(4)
Der Wirtschaftsrat wird für seine Sitzungen vom Vorsitzenden des Wirtschaftsrates schriftlich oder mündlich einberufen.
Er hat auch auf Antrag des Vorstandes oder von sich aus zusammenzutreten, wenn mindestens ein Drittel seiner Mitglieder dies wünscht.

(7)
Außer durch Tod erlischt die Funktion eines Wirtschaftsrats-Mitgliedes durch Rücktritt, durch Auflösung des Sponsorvertrages oder durch Reduktion der Sponsorsumme unter die in § 15 Abs 1 vorgesehene Grenze.
Wirtschaftsrat-Mitglieder, welche dreimal hintereinander den Wirtschaftsrat- Sitzungen fernbleiben, können vom Wirtschaftsrat ihres Mandates enthoben werden.

(8)
Mitglieder des Wirtschaftsrates können nur ordentliche Vereinsmitglieder sein.
Sie sind an keine Weisungen gebunden und üben ihre Tätigkeit in voller Unabhängigkeit zum allgemeinen Wohl des Vereins aus.

\subsection{§ 16
Aufgabenkreis des Wirtschaftsrates}

(1)
Der Wirtschaftsrat berät den Vorstand, indem er Empfehlungen und Schlussfolgerungen zu allen Angelegenheiten wirtschaftlicher Natur formuliert.
Insbesondere unterstützt er den Vorstand mit dem Ziel, die wirtschaftliche und finanzielle Grundlage des Vereins in und außerhalb Tirols zu verbreitern und zu vertiefen.

(2)
Der Wirtschaftsrat kann vom Vorstand und der Geschäftsführung jederzeit einen Bericht über die Angelegenheiten des Vereins verlangen.

(3)
Der Wirtschaftsrat kann die Bücher und Schriften des Vereins sowie die Vermögensgegenstände, namentlich die Vereinskasse und die Bestände an Wertpapieren und Waren, einsehen und prüfen, er kann damit auch einzelne Mitglieder oder für bestimmte Aufgaben besondere Sachverständige auf eigene Kosten beauftragen.

(4)
Der Voranschlag wird dem Wirtschaftsrat zur Kenntnis gebracht.

(5)
Der Wirtschaftsrat ist vor dem Abschluss von Verträgen mit Sponsoren im Wert von über € 50.000,-- anzuhören.

(6)
Der Wirtschaftsrat ist vor der Festlegung oder Änderung allgemeiner Grundsätze der Vereinspolitik anzuhören.

(7)
Dem Wirtschaftsrat sind der Rechnungsabschluss und der Rechenschaftsbericht vorzulegen.

(8)
Der Wirtschaftsrat hat der Generalversammlung über seine Tätigkeit zu berichten.

(10)
Zur Regelung der inneren Organisation wird vom Wirtschaftsrat unter Berücksichtigung der Vereinsstatuten eine Geschäftsordnung beschlossen.

\subsection{§ 17
Rechnungsprüfer}

(1)
Die beiden Rechnungsprüfer, die betriebswirtschaftliche Kenntnisse besitzen müssen, werden von der Generalversammlung für die Funktionsdauer des Vorstandes gewählt.
Eine Wiederwahl ist möglich.
Sie dürfen keinem Organ mit Ausnahme der Generalversammlung angehören, dessen Tätigkeit Gegenstand der Prüfung ist.

(2)
Es obliegt ihnen die laufende Geschäftskontrolle und die Überprüfung der gesamten Geldgebarungen sowie des Jahresabschlusses und der Ordnungsmäßigkeit der Rechnungslegung und der statutengemäßen Verwendung der Mittel.
Sie haben über das Ergebnis der Überprüfung der Generalversammlung zu berichten.

(3)
Nur die Rechnungsprüfer können in der Generalversammlung den Antrag auf Entlastung des Vorstandes stellen.

\subsection{§ 18
Das Schiedsgericht}

(1)
In allen aus dem Vereinsverhältnis entstehenden Streitigkeiten entscheidet das Schiedsgericht, dies ist eine Schlichtungseinrichtung im Sinne des VG 2002 und kein Schiedsgericht nach den §§ 577 ZPO.
Das Schiedsgericht setzt sich aus 5 ordentlichen Mitgliedern zusammen.
Es wird derart gebildet, dass jeder Streitteil über Aufforderung durch den Vorstand innerhalb von 14 Tagen dem Vorstand 2 ordentliche Mitglieder als Schiedsrichter namhaft macht.
Diese wählen mit Stimmenmehrheit einen Vorsitzenden.
Bei Stimmengleichheit entscheidet das Los.

(2)
Die Einberufung des Schiedsgerichts erfolgt durch den Vorsitzenden.
Das Schiedsgericht entscheidet, ohne an bestimmte Vorschriften gebunden zu sein, nach besten Wissen und Gewissen.
Es trifft seine Entscheidungen bei Anwesenheit aller seiner Mitglieder mit einfacher Stimmenmehrheit.
Die Entscheidungen sind vereinsintern endgültig.

(3)
Mitglieder, die sich in einer Streitigkeit aus dem Vereinsverhältnis nicht dem Schiedsgericht unterwerfen oder dessen Entscheidungen nicht anerkennen, können aus dem Verein ausgeschlossen werden.

\clearpage

\section{Drittes Hauptstück: Das demokratische Leben des FC Wacker Innsbruck}

\subsection{§ 19
Demokratieverständnis}

(1)
Organisation und Arbeitsweise des Mitgliedervereins "`FC Wacker Innsbruck"' beruhen auf dem Grundsatz der repräsentativen Demokratie.
Die gemeinschaftliche Willensbildung gründet auf der Selbstbestimmung seiner Mitglieder und verfolgt das Ziel der Selbstbestimmtheit des Vereins.

(2)
Der Verein achtet in seinem gesamten Handeln den Grundsatz der Gleichheit seiner Mitglieder.
Diese genießen ein gleiches Maß an Aufmerksamkeit seitens der Organe des Vereins.

(3)
Die Organe des Vereins geben den Mitgliedern in geeigneter Weise die Möglichkeit, ihre Ansichten mit ihnen auszutauschen und öffentlich bekannt zu geben.
Die Organe des Vereins pflegen insbesondere auf Vereinsabenden einen offenen, transparenten und regelmäßigen Dialog mit den Mitgliedern.
Die Entscheidungen werden so offen und so mitgliedernah wie möglich getroffen.

\subsection{§ 20
Wahl des Vorstandes}

(1)
Der Obmann und die sonstigen Mitglieder des Vorstandes werden auf der Generalversammlung in getrennten Wahlgängen bestimmt.
Zunächst der Obmann im Verfahren der Einzelwahl, sodann ein von ihm unter Angabe der Funktionen vorgeschlagenes Team sonstiger Vorstandsmitglieder im Verfahren der Blockwahl.
Die Mitglieder des Vorstands gelten erst dann als bestellt, wenn sowohl Obmann als auch Team gewählt sind.
Die Wahlen sind in jedem Fall geheim durchzuführen.

(2)
Das Wahlverfahren vor der Generalversammlung gliedert sich wie folgt:

\begin{enumerate}[label=\alph*)]
\item
Für die Funktion des Obmanns ist beim Vorstand, organisatorisch vertreten durch das Clubmanagement, ein schriftlicher Wahlvorschlag einzubringen.
Jedes Mitglied ist berechtigt, ein ordentliches Mitglied zu nominieren.
Der Wahlvorschlag hat den vollständigen Namen, Meldeanschrift, Geburtsdatum und Geburtsort sowie die eigenhändige Unterschrift sowohl des Vorgeschlagenen als auch des (der) Vorschlagenden zu enthalten.
Die Nominierungsfrist endet zwei Wochen vor der Generalversammlung.
\item
Der nominierte Obmannkandidat hat den Wahlvorschlag um sein Team der sonstigen Vorstandsmitglieder (Teamvorschlag) zu ergänzen.
Voraussetzung ist deren ordentliche Mitgliedschaft.
Der Ergänzungsvorschlag hat den vollständigen Namen, Meldeanschrift, Geburtsdatum und Geburtsort sowie die eigenhändige Unterschrift sowohl der Vorgeschlagenen als auch des Vorschlagenden zu enthalten.
Die Ergänzungsfrist endet eine Woche vor der Generalversammlung.
\item
Das Clubmanagement hat die Mitglieder mit Ablauf der jeweiligen Fristen über die eingebrachten Vorschläge zu informieren.
Liegen mit Ablauf der jeweiligen Fristen keine Vorschläge vor, so kann der Vorstand eine Nachfrist setzen.
\item
Eine Woche vor der Generalversammlung ist ein Vereinsabend durchzuführen.
Den durch Wahlvorschlag nominierten Kandidaten ist dabei Gelegenheit zur Präsentation einzuräumen.
Ab dem Vereinsabend kann der nominierte Obmannkandidat seinen Teamvorschlag bis zur Durchführung der Blockwahl auf der Generalversammlung jederzeit ändern.
\end{enumerate}

(3)
Das Wahlverfahren auf der Generalversammlung steht unter der Leitung des Vorstands und gliedert sich wie folgt:

\begin{enumerate}[label=\alph*)]
\item
Über den Obmann wird gemäß Absatz 1 persönlich abgestimmt.
Es gilt jener Kandidat als gewählt, der die Mehrheit der gültigen Stimmen erhält.
Stellt sich nur ein Kandidat der Wahl, so muss er die absolute Mehrheit der gültigen Stimmen auf sich vereinigen können.
\item
Über die sonstigen Mitglieder des Vorstandes (Teamvorschlag) wird gemäß Absatz 1 im Block abgestimmt.
Die Wahl ist als Abstimmung mit den Wahlmöglichkeiten „ja“ und „nein“ durchzuführen.
Der Teamvorschlag gilt als angenommen, wenn er die absolute Mehrheit der gültigen Stimmen erhält.
\item
Erhält der Teamvorschlag nicht die erforderliche Mehrheit, so kann der bereits gewählte Obmannkandidat diesen entweder
\begin{enumerate}[label=\arabic*]
\item
ändern und erneut zur Abstimmung gemäß Buchstabe b stellen oder
\item
von seiner Wahl zurücktreten.
\end{enumerate}
\item
Das Recht zur Änderung des Teamvorschlags gemäß § 20 Abs 3 lit c Z 1 kann nur einmal in Anspruch genommen werden.
Führt das Wahlverfahren auf der Generalversammlung zu keinem Ergebnis, so ist jeder Rechnungsprüfer verpflichtet, unverzüglich eine außerordentliche Generalversammlung zum Zwecke der Wahl eines Vorstandes einzuberufen.
Diese hat jedenfalls innerhalb einer Frist von 42 Tagen stattzufinden.
Das Wahlverfahren startet erneut.
\item
Der amtierende Vorstand bleibt so lange im Amt, bis ein neuer Vorstand gewählt ist.
\end{enumerate}

\subsection{§ 21
Wahl der Rechnungsprüfer}

Über die Rechnungsprüfer wird in persönlicher Wahl einzeln abgestimmt.
Es gilt jener Kandidat als gewählt, der die Mehrheit der gültigen Stimmen erhält.
Stellt sich nur ein Kandidat der Wahl, so muss er die absolute Mehrheit der gültigen Stimmen auf sich vereinigen können.

\clearpage

\section{Viertes Hauptstück: Schlussbestimmungen}

\subsection{§ 22
Begriffliches Verständnis}

Um Verständlichkeit und Lesbarkeit der vorliegenden Statuten zu gewährleisten, sind die Darstellung und Beschreibung der Personen und Funktionen sowohl in der männlichen als auch in der weiblichen Form zu verstehen.

\subsection{§ 23
Auflösung}

(1)
Die freiwillige Auflösung des Vereines kann nur in einer zu diesem Zweck einberufenen außerordentlichen Generalversammlung und nur mit einer Mehrheit von zwei Drittel der gültigen Stimmen beschlossen werden.

(2)
Diese Generalversammlung hat auch – sofern Vereinsvermögen vorhanden ist – über die Abwicklung zu beschließen.
Insbesondere hat sie einen Abwickler zu berufen und Beschluss darüber zu fassen, wem dieser das nach Abdeckung der Passiven verbleibende Vermögen zu übertragen hat.

(3)
Bei Auflösung des Vereins oder bei Wegfall des bisherigen begünstigten Vereinszwecks darf das allenfalls vorhandene Vereinsvermögen in keiner wie immer gearteten Form den Vereinsmitgliedern zugute kommen, sondern ist für sportliche gemeinnützige Zwecke im Sinne der §§ 34ff BAO zu verwenden.

\end{document}